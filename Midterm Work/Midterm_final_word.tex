% Options for packages loaded elsewhere
\PassOptionsToPackage{unicode}{hyperref}
\PassOptionsToPackage{hyphens}{url}
%
\documentclass[
  ignorenonframetext,
]{beamer}
\usepackage{pgfpages}
\setbeamertemplate{caption}[numbered]
\setbeamertemplate{caption label separator}{: }
\setbeamercolor{caption name}{fg=normal text.fg}
\beamertemplatenavigationsymbolsempty
% Prevent slide breaks in the middle of a paragraph
\widowpenalties 1 10000
\raggedbottom
\setbeamertemplate{part page}{
  \centering
  \begin{beamercolorbox}[sep=16pt,center]{part title}
    \usebeamerfont{part title}\insertpart\par
  \end{beamercolorbox}
}
\setbeamertemplate{section page}{
  \centering
  \begin{beamercolorbox}[sep=12pt,center]{part title}
    \usebeamerfont{section title}\insertsection\par
  \end{beamercolorbox}
}
\setbeamertemplate{subsection page}{
  \centering
  \begin{beamercolorbox}[sep=8pt,center]{part title}
    \usebeamerfont{subsection title}\insertsubsection\par
  \end{beamercolorbox}
}
\AtBeginPart{
  \frame{\partpage}
}
\AtBeginSection{
  \ifbibliography
  \else
    \frame{\sectionpage}
  \fi
}
\AtBeginSubsection{
  \frame{\subsectionpage}
}
\usepackage{amsmath,amssymb}
\usepackage{lmodern}
\usepackage{ifxetex,ifluatex}
\ifnum 0\ifxetex 1\fi\ifluatex 1\fi=0 % if pdftex
  \usepackage[T1]{fontenc}
  \usepackage[utf8]{inputenc}
  \usepackage{textcomp} % provide euro and other symbols
\else % if luatex or xetex
  \usepackage{unicode-math}
  \defaultfontfeatures{Scale=MatchLowercase}
  \defaultfontfeatures[\rmfamily]{Ligatures=TeX,Scale=1}
\fi
\usetheme[]{AnnArbor}
\usecolortheme{dolphin}
\usefonttheme{structurebold}
% Use upquote if available, for straight quotes in verbatim environments
\IfFileExists{upquote.sty}{\usepackage{upquote}}{}
\IfFileExists{microtype.sty}{% use microtype if available
  \usepackage[]{microtype}
  \UseMicrotypeSet[protrusion]{basicmath} % disable protrusion for tt fonts
}{}
\makeatletter
\@ifundefined{KOMAClassName}{% if non-KOMA class
  \IfFileExists{parskip.sty}{%
    \usepackage{parskip}
  }{% else
    \setlength{\parindent}{0pt}
    \setlength{\parskip}{6pt plus 2pt minus 1pt}}
}{% if KOMA class
  \KOMAoptions{parskip=half}}
\makeatother
\usepackage{xcolor}
\IfFileExists{xurl.sty}{\usepackage{xurl}}{} % add URL line breaks if available
\IfFileExists{bookmark.sty}{\usepackage{bookmark}}{\usepackage{hyperref}}
\hypersetup{
  pdftitle={IMDB Feature Film Analysis},
  pdfauthor={T2 Deep Learners: Yue Li, Shuting Cai, Mrunalini Devineni, Siddharth Das},
  hidelinks,
  pdfcreator={LaTeX via pandoc}}
\urlstyle{same} % disable monospaced font for URLs
\newif\ifbibliography
\usepackage{color}
\usepackage{fancyvrb}
\newcommand{\VerbBar}{|}
\newcommand{\VERB}{\Verb[commandchars=\\\{\}]}
\DefineVerbatimEnvironment{Highlighting}{Verbatim}{commandchars=\\\{\}}
% Add ',fontsize=\small' for more characters per line
\usepackage{framed}
\definecolor{shadecolor}{RGB}{248,248,248}
\newenvironment{Shaded}{\begin{snugshade}}{\end{snugshade}}
\newcommand{\AlertTok}[1]{\textcolor[rgb]{0.94,0.16,0.16}{#1}}
\newcommand{\AnnotationTok}[1]{\textcolor[rgb]{0.56,0.35,0.01}{\textbf{\textit{#1}}}}
\newcommand{\AttributeTok}[1]{\textcolor[rgb]{0.77,0.63,0.00}{#1}}
\newcommand{\BaseNTok}[1]{\textcolor[rgb]{0.00,0.00,0.81}{#1}}
\newcommand{\BuiltInTok}[1]{#1}
\newcommand{\CharTok}[1]{\textcolor[rgb]{0.31,0.60,0.02}{#1}}
\newcommand{\CommentTok}[1]{\textcolor[rgb]{0.56,0.35,0.01}{\textit{#1}}}
\newcommand{\CommentVarTok}[1]{\textcolor[rgb]{0.56,0.35,0.01}{\textbf{\textit{#1}}}}
\newcommand{\ConstantTok}[1]{\textcolor[rgb]{0.00,0.00,0.00}{#1}}
\newcommand{\ControlFlowTok}[1]{\textcolor[rgb]{0.13,0.29,0.53}{\textbf{#1}}}
\newcommand{\DataTypeTok}[1]{\textcolor[rgb]{0.13,0.29,0.53}{#1}}
\newcommand{\DecValTok}[1]{\textcolor[rgb]{0.00,0.00,0.81}{#1}}
\newcommand{\DocumentationTok}[1]{\textcolor[rgb]{0.56,0.35,0.01}{\textbf{\textit{#1}}}}
\newcommand{\ErrorTok}[1]{\textcolor[rgb]{0.64,0.00,0.00}{\textbf{#1}}}
\newcommand{\ExtensionTok}[1]{#1}
\newcommand{\FloatTok}[1]{\textcolor[rgb]{0.00,0.00,0.81}{#1}}
\newcommand{\FunctionTok}[1]{\textcolor[rgb]{0.00,0.00,0.00}{#1}}
\newcommand{\ImportTok}[1]{#1}
\newcommand{\InformationTok}[1]{\textcolor[rgb]{0.56,0.35,0.01}{\textbf{\textit{#1}}}}
\newcommand{\KeywordTok}[1]{\textcolor[rgb]{0.13,0.29,0.53}{\textbf{#1}}}
\newcommand{\NormalTok}[1]{#1}
\newcommand{\OperatorTok}[1]{\textcolor[rgb]{0.81,0.36,0.00}{\textbf{#1}}}
\newcommand{\OtherTok}[1]{\textcolor[rgb]{0.56,0.35,0.01}{#1}}
\newcommand{\PreprocessorTok}[1]{\textcolor[rgb]{0.56,0.35,0.01}{\textit{#1}}}
\newcommand{\RegionMarkerTok}[1]{#1}
\newcommand{\SpecialCharTok}[1]{\textcolor[rgb]{0.00,0.00,0.00}{#1}}
\newcommand{\SpecialStringTok}[1]{\textcolor[rgb]{0.31,0.60,0.02}{#1}}
\newcommand{\StringTok}[1]{\textcolor[rgb]{0.31,0.60,0.02}{#1}}
\newcommand{\VariableTok}[1]{\textcolor[rgb]{0.00,0.00,0.00}{#1}}
\newcommand{\VerbatimStringTok}[1]{\textcolor[rgb]{0.31,0.60,0.02}{#1}}
\newcommand{\WarningTok}[1]{\textcolor[rgb]{0.56,0.35,0.01}{\textbf{\textit{#1}}}}
\usepackage{graphicx}
\makeatletter
\def\maxwidth{\ifdim\Gin@nat@width>\linewidth\linewidth\else\Gin@nat@width\fi}
\def\maxheight{\ifdim\Gin@nat@height>\textheight\textheight\else\Gin@nat@height\fi}
\makeatother
% Scale images if necessary, so that they will not overflow the page
% margins by default, and it is still possible to overwrite the defaults
% using explicit options in \includegraphics[width, height, ...]{}
\setkeys{Gin}{width=\maxwidth,height=\maxheight,keepaspectratio}
% Set default figure placement to htbp
\makeatletter
\def\fps@figure{htbp}
\makeatother
\setlength{\emergencystretch}{3em} % prevent overfull lines
\providecommand{\tightlist}{%
  \setlength{\itemsep}{0pt}\setlength{\parskip}{0pt}}
\setcounter{secnumdepth}{-\maxdimen} % remove section numbering
\ifluatex
  \usepackage{selnolig}  % disable illegal ligatures
\fi

\title{IMDB Feature Film Analysis}
\author{T2 Deep Learners: Yue Li, Shuting Cai, Mrunalini Devineni,
Siddharth Das}
\date{2021-11-09}

\begin{document}
\frame{\titlepage}

\begin{frame}{Introduction}
\protect\hypertarget{introduction}{}
Movies are a great way to pass our leisure time. With the rapid
development of the Internet, film review aggregators such as IMDb and
Rotten Tomatoes have gradually become very popular. Hence, film reviews
are of great importance for most movie audiences and film production
companies. Generally, movies with a higher rating will obtain more
attention and do better at the box office. It is not uncommon for movie
reviewers to be hired by stakeholders, pretending to be ordinary
netizens, giving high movie ratings, and posting favorable comments on
the website for profit. In addition, some netizens give extreme movie
ratings based on their preference for certain directors or actors
instead of standing in neutrality. Such bogus movie ratings will mislead
the public and may cause adverse effects. This issue has aroused
considerable public attention, leading many researchers to start work on
spurious comment detection. At present, most of the work focuses on
judging whether a specific user or comment is reliable. In this project,
the approach of our team is to construct a rating system that analyzes
the average movie votes since it is the main factor for users to judge
the quality of the movie. Gathering initial insights on the overall
distribution of movie ratings might eliminate the influence of bogus
reviews on the average movie score, leading to an authentic movie rating
distribution. We also explore the factors and their interactions that
affect movie ratings. Our end goal is to build a prediction model to
help clients understand the trends in the film market to help them make
the right decisions on movie production.

Prior research on IMDB movie reviews includes movie review text
classification using sentiment analysis on a data set comprising 1,000
positive and 1,000 negative reviews. (Brownlee, 2020)

The bag-of-words feature extraction technique creates unigrams, bigrams,
and trigrams as a feature set and represents it as a vector. The Naive
Bayes algorithm is employed to categorize the movie reviews into
negative and positive reviews. The word2vec model summarizes movie
reviews by extracting features from classified movie review sentences,
and the semantic clustering technique clusters semantically related
review sentences. Different text features are employed to compute the
salience score of all review sentences in the cluster. (Khan etc., 2020)
\end{frame}

\begin{frame}[fragile]{Description of the Data}
\protect\hypertarget{description-of-the-data}{}
The dataset is from the Kaggle, divided into three separate CSV files.
It contains 22 variables of 85,855 movies spanning from 1894 to 2020
from multiple countries. There are also 297,705 instances of cast
information. The ``in development'' titles are not included in the files
and contain missing categories of data like the short plot on the main
page, awards, external reviews, parent's guide, synopsis, faqs, news,
etc. Additional features such as the production company, title groups,
adult titles, instant watch options like Amazon Prime or Netflix could
make the analysis much broader and ideal.

\begin{Shaded}
\begin{Highlighting}[]
\NormalTok{movies }\OtherTok{\textless{}{-}} \FunctionTok{read.csv}\NormalTok{(}\StringTok{\textquotesingle{}movies.csv\textquotesingle{}}\NormalTok{, }\AttributeTok{header=}\ConstantTok{TRUE}\NormalTok{)}
\NormalTok{drop }\OtherTok{\textless{}{-}} \FunctionTok{c}\NormalTok{(}\StringTok{\textquotesingle{}budget\textquotesingle{}}\NormalTok{,}\StringTok{\textquotesingle{}usa\_gross\_income\textquotesingle{}}\NormalTok{,}\StringTok{\textquotesingle{}worlwide\_gross\_income\textquotesingle{}}\NormalTok{,}\StringTok{\textquotesingle{}metascore\textquotesingle{}}\NormalTok{,}\StringTok{\textquotesingle{}reviews\_from\_users\textquotesingle{}}\NormalTok{,}
          \StringTok{\textquotesingle{}reviews\_from\_critics\textquotesingle{}}\NormalTok{,}\StringTok{\textquotesingle{}production\_company\textquotesingle{}}\NormalTok{,}\StringTok{\textquotesingle{}description\textquotesingle{}}\NormalTok{,}\StringTok{\textquotesingle{}writer\textquotesingle{}}\NormalTok{)}
\NormalTok{movies }\OtherTok{\textless{}{-}}\NormalTok{ movies[,}\SpecialCharTok{!}\FunctionTok{names}\NormalTok{(movies) }\SpecialCharTok{\%in\%}\NormalTok{ drop]}
\NormalTok{movies }\OtherTok{\textless{}{-}}\NormalTok{ movies[}\SpecialCharTok{!}\NormalTok{(movies}\SpecialCharTok{$}\NormalTok{country }\SpecialCharTok{==} \StringTok{""}\NormalTok{) }\SpecialCharTok{\&} \SpecialCharTok{!}\NormalTok{(movies}\SpecialCharTok{$}\NormalTok{language }\SpecialCharTok{==} \StringTok{""}\NormalTok{) }\SpecialCharTok{\&} \SpecialCharTok{!}\NormalTok{(movies}\SpecialCharTok{$}\NormalTok{director }\SpecialCharTok{==} \StringTok{""}\NormalTok{) }\SpecialCharTok{\&} \SpecialCharTok{!}\NormalTok{(movies}\SpecialCharTok{$}\NormalTok{actors }\SpecialCharTok{==} \StringTok{""}\NormalTok{),]}

\NormalTok{ratings }\OtherTok{\textless{}{-}} \FunctionTok{read.csv}\NormalTok{(}\StringTok{\textquotesingle{}ratings.csv\textquotesingle{}}\NormalTok{, }\AttributeTok{header=}\ConstantTok{TRUE}\NormalTok{)}
\NormalTok{drop }\OtherTok{\textless{}{-}} \FunctionTok{c}\NormalTok{(}\StringTok{\textquotesingle{}allgenders\_0age\_avg\_vote\textquotesingle{}}\NormalTok{, }\StringTok{\textquotesingle{}allgenders\_0age\_votes\textquotesingle{}}\NormalTok{, }\StringTok{\textquotesingle{}males\_0age\_avg\_vote\textquotesingle{}}\NormalTok{, }\StringTok{\textquotesingle{}males\_0age\_votes\textquotesingle{}}\NormalTok{,}
          \StringTok{\textquotesingle{}females\_0age\_avg\_vote\textquotesingle{}}\NormalTok{, }\StringTok{\textquotesingle{}females\_0age\_votes\textquotesingle{}}\NormalTok{, }\StringTok{\textquotesingle{}us\_voters\_rating\textquotesingle{}}\NormalTok{, }\StringTok{\textquotesingle{}us\_voters\_votes\textquotesingle{}}\NormalTok{, }
          \StringTok{\textquotesingle{}non\_us\_voters\_rating\textquotesingle{}}\NormalTok{, }\StringTok{\textquotesingle{}non\_us\_voters\_votes\textquotesingle{}}\NormalTok{)}
\NormalTok{ratings }\OtherTok{\textless{}{-}}\NormalTok{ ratings[,}\SpecialCharTok{!}\FunctionTok{names}\NormalTok{(ratings) }\SpecialCharTok{\%in\%}\NormalTok{ drop]}
\NormalTok{ratings }\OtherTok{\textless{}{-}} \FunctionTok{na.omit}\NormalTok{(ratings)}

\NormalTok{movie\_ratings }\OtherTok{\textless{}{-}} \FunctionTok{merge}\NormalTok{(movies, ratings, }\AttributeTok{by=}\StringTok{"imdb\_title\_id"}\NormalTok{ )}
\FunctionTok{str}\NormalTok{(movies)}
\end{Highlighting}
\end{Shaded}

\begin{verbatim}
## 'data.frame':    84856 obs. of  13 variables:
##  $ imdb_title_id : chr  "tt0000009" "tt0000574" "tt0002101" "tt0002130" ...
##  $ title         : chr  "Miss Jerry" "The Story of the Kelly Gang" "Cleopatra" "L'Inferno" ...
##  $ original_title: chr  "Miss Jerry" "The Story of the Kelly Gang" "Cleopatra" "L'Inferno" ...
##  $ year          : chr  "1894" "1906" "1912" "1911" ...
##  $ date_published: chr  "1894-10-09" "1906-12-26" "1912-11-13" "1911-03-06" ...
##  $ genre         : chr  "Romance" "Biography, Crime, Drama" "Drama, History" "Adventure, Drama, Fantasy" ...
##  $ duration      : int  45 70 100 68 60 85 120 55 121 54 ...
##  $ country       : chr  "USA" "Australia" "USA" "Italy" ...
##  $ language      : chr  "None" "None" "English" "Italian" ...
##  $ director      : chr  "Alexander Black" "Charles Tait" "Charles L. Gaskill" "Francesco Bertolini, Adolfo Padovan" ...
##  $ actors        : chr  "Blanche Bayliss, William Courtenay, Chauncey Depew" "Elizabeth Tait, John Tait, Norman Campbell, Bella Cola, Will Coyne, Sam Crewes, Jack Ennis, John Forde, Vera Li"| __truncated__ "Helen Gardner, Pearl Sindelar, Miss Fielding, Miss Robson, Helene Costello, Charles Sindelar, Mr. Howard, James"| __truncated__ "Salvatore Papa, Arturo Pirovano, Giuseppe de Liguoro, Pier Delle Vigne, Augusto Milla, Attilio Motta, Emilise Beretta" ...
##  $ avg_vote      : num  5.9 6.1 5.2 7 5.7 6.8 6.2 5.5 6.6 7 ...
##  $ votes         : int  154 589 446 2237 484 753 273 225 331 1944 ...
\end{verbatim}
\end{frame}

\begin{frame}[fragile]{Independent Variables EDA}
\protect\hypertarget{independent-variables-eda}{}
\begin{block}{Movie Rating Distribution}
\protect\hypertarget{movie-rating-distribution}{}
\begin{Shaded}
\begin{Highlighting}[]
\NormalTok{movie\_genre\_1 }\OtherTok{\textless{}{-}}\NormalTok{ movie\_ratings[,}\FunctionTok{c}\NormalTok{(}\StringTok{"genre"}\NormalTok{, }\StringTok{"avg\_vote"}\NormalTok{)]}

\NormalTok{data\_mg }\OtherTok{\textless{}{-}} \FunctionTok{aggregate}\NormalTok{(movie\_genre\_1}\SpecialCharTok{$}\NormalTok{avg\_vote, }\FunctionTok{list}\NormalTok{(movie\_genre\_1}\SpecialCharTok{$}\NormalTok{genre), }\AttributeTok{FUN=}\NormalTok{mean, }\AttributeTok{sort =} \ConstantTok{FALSE}\NormalTok{)}
\FunctionTok{names}\NormalTok{(data\_mg)[}\FunctionTok{names}\NormalTok{(data\_mg) }\SpecialCharTok{==} \StringTok{"Group.1"}\NormalTok{] }\OtherTok{\textless{}{-}} \StringTok{"genre"}
\CommentTok{\#count\_mg \textless{}{-} movie\_genre\_1\%\textgreater{}\%count(genre)}
\FunctionTok{names}\NormalTok{(data\_mg)[}\FunctionTok{names}\NormalTok{(data\_mg) }\SpecialCharTok{==} \StringTok{"x"}\NormalTok{] }\OtherTok{\textless{}{-}} \StringTok{"rating"}
\FunctionTok{names}\NormalTok{(data\_mg)[}\FunctionTok{names}\NormalTok{(data\_mg) }\SpecialCharTok{==}\StringTok{"n"}\NormalTok{] }\OtherTok{\textless{}{-}} \StringTok{"count"}
\NormalTok{data\_mg}\SpecialCharTok{$}\NormalTok{genre }\OtherTok{\textless{}{-}} \FunctionTok{as.factor}\NormalTok{(data\_mg}\SpecialCharTok{$}\NormalTok{genre)}

\CommentTok{\# summary of dataset}
\FunctionTok{summary}\NormalTok{(data\_mg}\SpecialCharTok{$}\NormalTok{rating)}
\end{Highlighting}
\end{Shaded}

\begin{verbatim}
##    Min. 1st Qu.  Median    Mean 3rd Qu.    Max. 
##    1.50    5.47    6.05    5.92    6.50    8.70
\end{verbatim}

\begin{Shaded}
\begin{Highlighting}[]
\CommentTok{\# plot histogram plot}
\FunctionTok{hist}\NormalTok{(data\_mg}\SpecialCharTok{$}\NormalTok{rating, }\AttributeTok{main =} \StringTok{"Histogram of the movie rating"}\NormalTok{, }\AttributeTok{xlab=}\StringTok{"Movie rating"}\NormalTok{, }\AttributeTok{col=}\StringTok{"\#66C2A5"}\NormalTok{, }\AttributeTok{breaks =} \DecValTok{15}\NormalTok{)}
\end{Highlighting}
\end{Shaded}

\includegraphics{Midterm_final_word_files/figure-beamer/movie_rating_\&_qq_plot-1.pdf}

\begin{Shaded}
\begin{Highlighting}[]
\CommentTok{\# plot qqplot}
\FunctionTok{qqnorm}\NormalTok{(data\_mg}\SpecialCharTok{$}\NormalTok{rating, }\AttributeTok{main =} \StringTok{"Q{-}Q plot for movie rating"}\NormalTok{)}
\FunctionTok{qqline}\NormalTok{(data\_mg}\SpecialCharTok{$}\NormalTok{rating)}
\end{Highlighting}
\end{Shaded}

\includegraphics{Midterm_final_word_files/figure-beamer/movie_rating_\&_qq_plot-2.pdf}

\begin{Shaded}
\begin{Highlighting}[]
\CommentTok{\#remove outliers}
\NormalTok{outliers }\OtherTok{\textless{}{-}} \FunctionTok{unique}\NormalTok{(}\FunctionTok{boxplot}\NormalTok{(data\_mg}\SpecialCharTok{$}\NormalTok{rating, }\AttributeTok{plot=}\ConstantTok{FALSE}\NormalTok{)}\SpecialCharTok{$}\NormalTok{out)}
\NormalTok{data\_mg }\OtherTok{\textless{}{-}}\NormalTok{ data\_mg[}\SpecialCharTok{{-}}\FunctionTok{which}\NormalTok{(data\_mg}\SpecialCharTok{$}\NormalTok{rating }\SpecialCharTok{\%in\%}\NormalTok{ outliers),]}
\end{Highlighting}
\end{Shaded}
\end{block}

\begin{block}{Movie Votes Distribution}
\protect\hypertarget{movie-votes-distribution}{}
\begin{Shaded}
\begin{Highlighting}[]
\NormalTok{movie\_genre\_v }\OtherTok{\textless{}{-}}\NormalTok{ movie\_ratings[,}\FunctionTok{c}\NormalTok{(}\StringTok{"genre"}\NormalTok{, }\StringTok{"votes"}\NormalTok{)]}
\NormalTok{data\_v }\OtherTok{\textless{}{-}} \FunctionTok{aggregate}\NormalTok{(movie\_genre\_v}\SpecialCharTok{$}\NormalTok{votes, }\FunctionTok{list}\NormalTok{(movie\_genre\_v}\SpecialCharTok{$}\NormalTok{genre), }\AttributeTok{FUN=}\NormalTok{sum, }\AttributeTok{sort =} \ConstantTok{FALSE}\NormalTok{)}
\FunctionTok{names}\NormalTok{(data\_v)[}\FunctionTok{names}\NormalTok{(data\_v) }\SpecialCharTok{==} \StringTok{"Group.1"}\NormalTok{] }\OtherTok{\textless{}{-}} \StringTok{"genre"}
\FunctionTok{names}\NormalTok{(data\_v)[}\FunctionTok{names}\NormalTok{(data\_v) }\SpecialCharTok{==} \StringTok{"x"}\NormalTok{] }\OtherTok{\textless{}{-}} \StringTok{"movie\_votes"}
\FunctionTok{names}\NormalTok{(data\_v)[}\FunctionTok{names}\NormalTok{(data\_v) }\SpecialCharTok{==}\StringTok{"n"}\NormalTok{] }\OtherTok{\textless{}{-}} \StringTok{"count"}
\NormalTok{data\_v}\SpecialCharTok{$}\NormalTok{genre }\OtherTok{\textless{}{-}} \FunctionTok{as.factor}\NormalTok{(data\_v}\SpecialCharTok{$}\NormalTok{genre)}

\CommentTok{\# summary of dataset}
\FunctionTok{summary}\NormalTok{(data\_v}\SpecialCharTok{$}\NormalTok{movie\_votes)}
\end{Highlighting}
\end{Shaded}

\begin{verbatim}
##     Min.  1st Qu.   Median     Mean  3rd Qu.     Max. 
##      101      853     4710   679217    98043 44471670
\end{verbatim}

\begin{Shaded}
\begin{Highlighting}[]
\CommentTok{\# plot histogram plot}
\FunctionTok{hist}\NormalTok{(data\_v}\SpecialCharTok{$}\NormalTok{movie\_votes, }\AttributeTok{main =} \StringTok{"Histogram of the movie votes"}\NormalTok{, }\AttributeTok{xlab=}\StringTok{"Movie votes"}\NormalTok{, }\AttributeTok{col=}\StringTok{"\#FC8D62"}\NormalTok{, }\AttributeTok{breaks =} \DecValTok{15}\NormalTok{)}
\end{Highlighting}
\end{Shaded}

\includegraphics{Midterm_final_word_files/figure-beamer/movie votes \& qq plot-1.pdf}

\begin{Shaded}
\begin{Highlighting}[]
\CommentTok{\# plot qq plot}
\FunctionTok{qqnorm}\NormalTok{(data\_v}\SpecialCharTok{$}\NormalTok{movie\_votes, }\AttributeTok{main =} \StringTok{"Q{-}Q plot for movie votes"}\NormalTok{)}
\FunctionTok{qqline}\NormalTok{(data\_v}\SpecialCharTok{$}\NormalTok{movie\_votes)}
\end{Highlighting}
\end{Shaded}

\includegraphics{Midterm_final_word_files/figure-beamer/movie votes \& qq plot-2.pdf}

\begin{Shaded}
\begin{Highlighting}[]
\CommentTok{\#remove outliers}
\NormalTok{outliers }\OtherTok{\textless{}{-}} \FunctionTok{unique}\NormalTok{(}\FunctionTok{boxplot}\NormalTok{(data\_v}\SpecialCharTok{$}\NormalTok{movie\_votes, }\AttributeTok{plot=}\ConstantTok{FALSE}\NormalTok{)}\SpecialCharTok{$}\NormalTok{out)}
\NormalTok{data\_v }\OtherTok{\textless{}{-}}\NormalTok{ data\_v[}\SpecialCharTok{{-}}\FunctionTok{which}\NormalTok{(data\_v}\SpecialCharTok{$}\NormalTok{movie\_votes }\SpecialCharTok{\%in\%}\NormalTok{ outliers),]}
\end{Highlighting}
\end{Shaded}
\end{block}

\begin{block}{Movie Genres Distribution}
\protect\hypertarget{movie-genres-distribution}{}
\begin{Shaded}
\begin{Highlighting}[]
\CommentTok{\# plot ggplot for rating}

\NormalTok{data\_mg }\OtherTok{\textless{}{-}} \FunctionTok{subset}\NormalTok{(data\_mg, rating}\SpecialCharTok{\textgreater{}}\FloatTok{7.5}\NormalTok{)}
\FunctionTok{ggplot}\NormalTok{(}\AttributeTok{data=}\NormalTok{data\_mg, }\FunctionTok{aes}\NormalTok{(}\AttributeTok{x=}\FunctionTok{reorder}\NormalTok{(genre, rating), }\AttributeTok{y=}\NormalTok{rating)) }\SpecialCharTok{+} 
  \FunctionTok{geom\_bar}\NormalTok{(}\AttributeTok{stat =} \StringTok{"identity"}\NormalTok{, }\AttributeTok{position=}\StringTok{"dodge"}\NormalTok{,}\AttributeTok{alpha=}\NormalTok{.}\DecValTok{8}\NormalTok{, }\AttributeTok{fill =} \StringTok{"\#66C2A5"}\NormalTok{) }\SpecialCharTok{+} 
    \FunctionTok{scale\_fill\_fermenter}\NormalTok{(}\AttributeTok{palette =} \StringTok{"Set2"}\NormalTok{) }\SpecialCharTok{+} 
      \FunctionTok{xlab}\NormalTok{(}\StringTok{"Movie genre"}\NormalTok{) }\SpecialCharTok{+} 
        \FunctionTok{ylab}\NormalTok{(}\StringTok{"Movie Rating"}\NormalTok{)}\SpecialCharTok{+} 
          \FunctionTok{labs}\NormalTok{(}\AttributeTok{title=}\StringTok{\textquotesingle{}The rating of various movie genres (rating \textgreater{}= 7.5)\textquotesingle{}}\NormalTok{) }\SpecialCharTok{+} 
            \FunctionTok{coord\_flip}\NormalTok{() }\SpecialCharTok{+} 
              \FunctionTok{theme}\NormalTok{(}\AttributeTok{text =} \FunctionTok{element\_text}\NormalTok{(}\AttributeTok{size=}\DecValTok{10}\NormalTok{), }\AttributeTok{legend.position=}\StringTok{"right"}\NormalTok{, }\AttributeTok{plot.title =} \FunctionTok{element\_text}\NormalTok{(}\AttributeTok{size=}\DecValTok{15}\NormalTok{))}
\end{Highlighting}
\end{Shaded}

\includegraphics{Midterm_final_word_files/figure-beamer/Genre distribution-1.pdf}

\begin{Shaded}
\begin{Highlighting}[]
\CommentTok{\# ggplot for movie votes}
\NormalTok{data\_v }\OtherTok{\textless{}{-}} \FunctionTok{subset}\NormalTok{(data\_v, movie\_votes }\SpecialCharTok{\textgreater{}} \DecValTok{200000}\NormalTok{)}
\FunctionTok{ggplot}\NormalTok{(}\AttributeTok{data=}\NormalTok{data\_v, }\FunctionTok{aes}\NormalTok{(}\AttributeTok{x=}\FunctionTok{reorder}\NormalTok{(genre, movie\_votes), }\AttributeTok{y=}\NormalTok{movie\_votes)) }\SpecialCharTok{+} 
  \FunctionTok{geom\_bar}\NormalTok{(}\AttributeTok{stat =} \StringTok{"identity"}\NormalTok{, }\AttributeTok{alpha=}\NormalTok{.}\DecValTok{7}\NormalTok{,}\AttributeTok{fill =} \StringTok{"\#FC8D62"}\NormalTok{) }\SpecialCharTok{+} 
    \FunctionTok{scale\_fill\_fermenter}\NormalTok{(}\AttributeTok{palette =} \StringTok{"Set2"}\NormalTok{) }\SpecialCharTok{+} 
      \FunctionTok{xlab}\NormalTok{(}\StringTok{"Movie genre"}\NormalTok{) }\SpecialCharTok{+} \FunctionTok{ylab}\NormalTok{(}\StringTok{"Movie Votes"}\NormalTok{)}\SpecialCharTok{+} 
        \FunctionTok{labs}\NormalTok{(}\AttributeTok{title=}\StringTok{\textquotesingle{}The sum of movie votes in movie genres (votes \textgreater{}= 200k)\textquotesingle{}}\NormalTok{) }\SpecialCharTok{+} 
          \FunctionTok{coord\_flip}\NormalTok{()}\SpecialCharTok{+}
            \FunctionTok{theme}\NormalTok{(}\AttributeTok{axis.text =} \FunctionTok{element\_text}\NormalTok{(}\AttributeTok{size=}\DecValTok{8}\NormalTok{), }\AttributeTok{legend.position=}\StringTok{"right"}\NormalTok{, }\AttributeTok{plot.title =} \FunctionTok{element\_text}\NormalTok{(}\AttributeTok{size=}\DecValTok{15}\NormalTok{))}
\end{Highlighting}
\end{Shaded}

\includegraphics{Midterm_final_word_files/figure-beamer/Genre distribution-2.pdf}
\end{block}

\begin{block}{Year-Wise Movie Distribution}
\protect\hypertarget{year-wise-movie-distribution}{}
\begin{Shaded}
\begin{Highlighting}[]
\CommentTok{\# average vote distribution}
\NormalTok{movies}\OtherTok{\textless{}{-}}\NormalTok{movies[}\SpecialCharTok{!}\NormalTok{movies}\SpecialCharTok{$}\NormalTok{year}\SpecialCharTok{==}\StringTok{\textquotesingle{}TV Movie 2019\textquotesingle{}} \SpecialCharTok{\&} \SpecialCharTok{!}\NormalTok{movies}\SpecialCharTok{$}\NormalTok{year}\SpecialCharTok{==}\StringTok{\textquotesingle{}2020\textquotesingle{}}\NormalTok{,]}

\CommentTok{\# A brief overview of the dataset: the minimum vote is 1.000,while the maximum vote is 9.900.}
\CommentTok{\# the mean of the vote is 5.902, the median is 6.100.}


\CommentTok{\# movies year{-}numbers distribution}
\NormalTok{years\_numbers }\OtherTok{\textless{}{-}}\NormalTok{ movies[,}\FunctionTok{c}\NormalTok{(}\StringTok{\textquotesingle{}imdb\_title\_id\textquotesingle{}}\NormalTok{, }\StringTok{\textquotesingle{}year\textquotesingle{}}\NormalTok{)]}
\NormalTok{years\_num }\OtherTok{\textless{}{-}}\NormalTok{ years\_numbers }\SpecialCharTok{\%\textgreater{}\%} \FunctionTok{count}\NormalTok{(year,}\AttributeTok{sort =} \ConstantTok{TRUE}\NormalTok{)}
\FunctionTok{colnames}\NormalTok{(years\_num)}\OtherTok{\textless{}{-}}\FunctionTok{c}\NormalTok{(}\StringTok{\textquotesingle{}Year\textquotesingle{}}\NormalTok{,}\StringTok{\textquotesingle{}Movie\_numbers\textquotesingle{}}\NormalTok{)}
\NormalTok{years\_num}\SpecialCharTok{$}\NormalTok{Year }\OtherTok{\textless{}{-}} \FunctionTok{as.numeric}\NormalTok{(years\_num}\SpecialCharTok{$}\NormalTok{Year)}
\NormalTok{df1 }\OtherTok{\textless{}{-}}\FunctionTok{arrange}\NormalTok{(years\_num, }\SpecialCharTok{{-}}\NormalTok{Movie\_numbers)}
\FunctionTok{head}\NormalTok{(df1)}
\end{Highlighting}
\end{Shaded}

\begin{verbatim}
##   Year Movie_numbers
## 1 2017          3246
## 2 2018          3158
## 3 2016          3076
## 4 2015          2939
## 5 2014          2900
## 6 2013          2763
\end{verbatim}

\begin{Shaded}
\begin{Highlighting}[]
\FunctionTok{tail}\NormalTok{(df1)}
\end{Highlighting}
\end{Shaded}

\begin{verbatim}
##     Year Movie_numbers
## 106 1917            14
## 107 1913            12
## 108 1912             4
## 109 1911             2
## 110 1894             1
## 111 1906             1
\end{verbatim}

\begin{Shaded}
\begin{Highlighting}[]
\FunctionTok{ggplot}\NormalTok{(years\_num,}\FunctionTok{aes}\NormalTok{(}\AttributeTok{x=}\NormalTok{Year,}\AttributeTok{y=}\NormalTok{Movie\_numbers)) }\SpecialCharTok{+} 
  \FunctionTok{geom\_bar}\NormalTok{(}\AttributeTok{stat=}\StringTok{"identity"}\NormalTok{, }\AttributeTok{fill=}\StringTok{\textquotesingle{}\#66C2A5\textquotesingle{}}\NormalTok{, }\AttributeTok{alpha=}\NormalTok{.}\DecValTok{8}\NormalTok{, }\AttributeTok{width=}\DecValTok{1}\NormalTok{) }\SpecialCharTok{+} 
    \FunctionTok{labs}\NormalTok{(}\AttributeTok{title=}\NormalTok{(}\StringTok{\textquotesingle{}Count of movies each year\textquotesingle{}}\NormalTok{)) }\SpecialCharTok{+} 
      \FunctionTok{scale\_fill\_fermenter}\NormalTok{(}\AttributeTok{palette =} \StringTok{"Set2"}\NormalTok{) }\SpecialCharTok{+} 
        \FunctionTok{ylab}\NormalTok{(}\StringTok{\textquotesingle{}Count\textquotesingle{}}\NormalTok{) }\SpecialCharTok{+}
          \FunctionTok{xlab}\NormalTok{(}\StringTok{\textquotesingle{}Year\textquotesingle{}}\NormalTok{) }\SpecialCharTok{+}
            \FunctionTok{coord\_flip}\NormalTok{() }\SpecialCharTok{+}
              \FunctionTok{scale\_x\_continuous}\NormalTok{(}\AttributeTok{breaks=}\FunctionTok{seq}\NormalTok{(}\DecValTok{1900}\NormalTok{, }\DecValTok{2020}\NormalTok{, }\DecValTok{10}\NormalTok{))}
\end{Highlighting}
\end{Shaded}

\includegraphics{Midterm_final_word_files/figure-beamer/EDA_year_movie_count-1.pdf}
\end{block}

\begin{block}{Average Votes vs.~Year}
\protect\hypertarget{average-votes-vs.-year}{}
\begin{Shaded}
\begin{Highlighting}[]
\CommentTok{\# movies year{-}rating distribution}
\NormalTok{years\_rating }\OtherTok{\textless{}{-}} \FunctionTok{aggregate}\NormalTok{(movies}\SpecialCharTok{$}\NormalTok{avg\_vote, }\AttributeTok{by =} \FunctionTok{list}\NormalTok{(movies}\SpecialCharTok{$}\NormalTok{year),}\AttributeTok{FUN=}\NormalTok{mean)}
\FunctionTok{colnames}\NormalTok{(years\_rating) }\OtherTok{\textless{}{-}} \FunctionTok{c}\NormalTok{(}\StringTok{\textquotesingle{}Year\textquotesingle{}}\NormalTok{,}\StringTok{\textquotesingle{}Rating\textquotesingle{}}\NormalTok{)}
\NormalTok{years\_rating}\SpecialCharTok{$}\NormalTok{Year }\OtherTok{\textless{}{-}} \FunctionTok{as.numeric}\NormalTok{(years\_rating}\SpecialCharTok{$}\NormalTok{Year)}
\NormalTok{df2 }\OtherTok{\textless{}{-}}\FunctionTok{arrange}\NormalTok{(years\_rating, }\SpecialCharTok{{-}}\NormalTok{Rating)}
\FunctionTok{head}\NormalTok{(df2)}
\end{Highlighting}
\end{Shaded}

\begin{verbatim}
##   Year Rating
## 1 1928   6.86
## 2 1921   6.76
## 3 1923   6.75
## 4 1924   6.69
## 5 1926   6.68
## 6 1911   6.60
\end{verbatim}

\begin{Shaded}
\begin{Highlighting}[]
\FunctionTok{tail}\NormalTok{(df2)}
\end{Highlighting}
\end{Shaded}

\begin{verbatim}
##     Year Rating
## 106 2009   5.64
## 107 2010   5.63
## 108 2012   5.63
## 109 2015   5.62
## 110 2011   5.62
## 111 2013   5.62
\end{verbatim}

\begin{Shaded}
\begin{Highlighting}[]
\FunctionTok{ggplot}\NormalTok{(}\AttributeTok{data=}\NormalTok{years\_rating,}\FunctionTok{aes}\NormalTok{(}\AttributeTok{x=}\NormalTok{Year,}\AttributeTok{y=}\NormalTok{Rating)) }\SpecialCharTok{+}
  \FunctionTok{geom\_bar}\NormalTok{(}\AttributeTok{stat=}\StringTok{"identity"}\NormalTok{, }\AttributeTok{fill=}\StringTok{\textquotesingle{}\#FC8D62\textquotesingle{}}\NormalTok{, }\AttributeTok{alpha=}\NormalTok{.}\DecValTok{8}\NormalTok{, }\AttributeTok{width=}\DecValTok{1}\NormalTok{) }\SpecialCharTok{+}
    \FunctionTok{scale\_fill\_fermenter}\NormalTok{(}\AttributeTok{palette =} \StringTok{"Set2"}\NormalTok{) }\SpecialCharTok{+} 
      \FunctionTok{labs}\NormalTok{(}\AttributeTok{title=}\NormalTok{(}\StringTok{\textquotesingle{}Average Rating every Year\textquotesingle{}}\NormalTok{)) }\SpecialCharTok{+} 
        \FunctionTok{ylab}\NormalTok{(}\StringTok{\textquotesingle{}Count\textquotesingle{}}\NormalTok{) }\SpecialCharTok{+}
          \FunctionTok{xlab}\NormalTok{(}\StringTok{\textquotesingle{}Year\textquotesingle{}}\NormalTok{) }\SpecialCharTok{+}
            \FunctionTok{coord\_flip}\NormalTok{() }\SpecialCharTok{+}
              \FunctionTok{scale\_x\_continuous}\NormalTok{(}\AttributeTok{breaks=}\FunctionTok{seq}\NormalTok{(}\DecValTok{1900}\NormalTok{, }\DecValTok{2020}\NormalTok{, }\DecValTok{10}\NormalTok{))}
\end{Highlighting}
\end{Shaded}

\includegraphics{Midterm_final_word_files/figure-beamer/EDA_year_avg_votes-1.pdf}
\end{block}

\begin{block}{Number of Votes vs.~Year}
\protect\hypertarget{number-of-votes-vs.-year}{}
\begin{Shaded}
\begin{Highlighting}[]
\CommentTok{\# movies vote distribution}
\NormalTok{years\_vote }\OtherTok{\textless{}{-}} \FunctionTok{aggregate}\NormalTok{(movies}\SpecialCharTok{$}\NormalTok{votes, }\AttributeTok{by =} \FunctionTok{list}\NormalTok{(}\AttributeTok{director =}\NormalTok{ movies}\SpecialCharTok{$}\NormalTok{year),}\AttributeTok{FUN=}\NormalTok{sum)}
\FunctionTok{colnames}\NormalTok{(years\_vote) }\OtherTok{\textless{}{-}} \FunctionTok{c}\NormalTok{(}\StringTok{\textquotesingle{}Year\textquotesingle{}}\NormalTok{,}\StringTok{\textquotesingle{}Votes\textquotesingle{}}\NormalTok{)}
\NormalTok{years\_vote}\SpecialCharTok{$}\NormalTok{Year }\OtherTok{\textless{}{-}} \FunctionTok{as.numeric}\NormalTok{(years\_vote}\SpecialCharTok{$}\NormalTok{Year)}
\NormalTok{df3 }\OtherTok{\textless{}{-}}\FunctionTok{arrange}\NormalTok{(years\_vote, }\SpecialCharTok{{-}}\NormalTok{Votes)}
\FunctionTok{head}\NormalTok{(df3)}
\end{Highlighting}
\end{Shaded}

\begin{verbatim}
##   Year    Votes
## 1 2013 37209029
## 2 2014 36190145
## 3 2011 33877671
## 4 2012 33740537
## 5 2008 31621851
## 6 2010 31496239
\end{verbatim}

\begin{Shaded}
\begin{Highlighting}[]
\FunctionTok{tail}\NormalTok{(df3)}
\end{Highlighting}
\end{Shaded}

\begin{verbatim}
##     Year Votes
## 106 1913  8880
## 107 1917  7147
## 108 1911  2399
## 109 1912  1460
## 110 1906   589
## 111 1894   154
\end{verbatim}

\begin{Shaded}
\begin{Highlighting}[]
\FunctionTok{ggplot}\NormalTok{(}\AttributeTok{data=}\NormalTok{years\_vote, }\FunctionTok{aes}\NormalTok{(}\AttributeTok{x=}\NormalTok{Year, }\AttributeTok{y=}\NormalTok{Votes)) }\SpecialCharTok{+}
  \FunctionTok{geom\_bar}\NormalTok{(}\AttributeTok{stat=}\StringTok{"identity"}\NormalTok{, }\AttributeTok{fill=}\StringTok{\textquotesingle{}\#66C2A5\textquotesingle{}}\NormalTok{, }\AttributeTok{alpha=}\NormalTok{.}\DecValTok{8}\NormalTok{, }\AttributeTok{width=}\DecValTok{1}\NormalTok{) }\SpecialCharTok{+}
    \FunctionTok{scale\_fill\_fermenter}\NormalTok{(}\AttributeTok{palette =} \StringTok{"Set2"}\NormalTok{) }\SpecialCharTok{+} 
      \FunctionTok{labs}\NormalTok{(}\AttributeTok{title=}\NormalTok{(}\StringTok{\textquotesingle{}Total Votes every Year\textquotesingle{}}\NormalTok{)) }\SpecialCharTok{+} 
        \FunctionTok{ylab}\NormalTok{(}\StringTok{\textquotesingle{}Votes\textquotesingle{}}\NormalTok{) }\SpecialCharTok{+}
          \FunctionTok{xlab}\NormalTok{(}\StringTok{\textquotesingle{}Year\textquotesingle{}}\NormalTok{) }\SpecialCharTok{+}
            \FunctionTok{coord\_flip}\NormalTok{() }\SpecialCharTok{+}
              \FunctionTok{scale\_x\_continuous}\NormalTok{(}\AttributeTok{breaks=}\FunctionTok{seq}\NormalTok{(}\DecValTok{1900}\NormalTok{, }\DecValTok{2020}\NormalTok{, }\DecValTok{10}\NormalTok{))}
\end{Highlighting}
\end{Shaded}

\includegraphics{Midterm_final_word_files/figure-beamer/EDA_year_votes-1.pdf}
\end{block}

\begin{block}{Duration of Movies vs.~Year}
\protect\hypertarget{duration-of-movies-vs.-year}{}
\begin{Shaded}
\begin{Highlighting}[]
\CommentTok{\# movies duration{-}year distribution}
\NormalTok{years\_duration }\OtherTok{\textless{}{-}} \FunctionTok{aggregate}\NormalTok{(movies}\SpecialCharTok{$}\NormalTok{duration, }\AttributeTok{by =} \FunctionTok{list}\NormalTok{(}\AttributeTok{director =}\NormalTok{ movies}\SpecialCharTok{$}\NormalTok{year),}\AttributeTok{FUN=}\NormalTok{mean)}
\FunctionTok{colnames}\NormalTok{(years\_duration) }\OtherTok{\textless{}{-}} \FunctionTok{c}\NormalTok{(}\StringTok{\textquotesingle{}Year\textquotesingle{}}\NormalTok{,}\StringTok{\textquotesingle{}Duration\textquotesingle{}}\NormalTok{)}
\NormalTok{years\_duration}\SpecialCharTok{$}\NormalTok{Year }\OtherTok{\textless{}{-}} \FunctionTok{as.numeric}\NormalTok{(years\_duration}\SpecialCharTok{$}\NormalTok{Year)}
\NormalTok{df4 }\OtherTok{\textless{}{-}}\FunctionTok{arrange}\NormalTok{(years\_duration, }\SpecialCharTok{{-}}\NormalTok{Duration)}
\FunctionTok{head}\NormalTok{(df4)}
\end{Highlighting}
\end{Shaded}

\begin{verbatim}
##   Year Duration
## 1 2019      105
## 2 1993      105
## 3 1992      105
## 4 2003      104
## 5 1980      104
## 6 1994      104
\end{verbatim}

\begin{Shaded}
\begin{Highlighting}[]
\FunctionTok{tail}\NormalTok{(df4)}
\end{Highlighting}
\end{Shaded}

\begin{verbatim}
##     Year Duration
## 106 1911     74.0
## 107 1920     74.0
## 108 1906     70.0
## 109 1917     67.6
## 110 1912     65.0
## 111 1894     45.0
\end{verbatim}

\begin{Shaded}
\begin{Highlighting}[]
\FunctionTok{ggplot}\NormalTok{(}\AttributeTok{data=}\NormalTok{years\_duration, }\FunctionTok{aes}\NormalTok{(}\AttributeTok{x=}\NormalTok{Year, }\AttributeTok{y=}\NormalTok{Duration)) }\SpecialCharTok{+}
  \FunctionTok{geom\_bar}\NormalTok{(}\AttributeTok{stat=}\StringTok{"identity"}\NormalTok{, }\AttributeTok{fill=}\StringTok{\textquotesingle{}\#FC8D62\textquotesingle{}}\NormalTok{, }\AttributeTok{alpha=}\NormalTok{.}\DecValTok{8}\NormalTok{, }\AttributeTok{width=}\DecValTok{1}\NormalTok{) }\SpecialCharTok{+}
    \FunctionTok{scale\_fill\_fermenter}\NormalTok{(}\AttributeTok{palette =} \StringTok{"Set2"}\NormalTok{) }\SpecialCharTok{+} 
      \FunctionTok{labs}\NormalTok{(}\AttributeTok{title=}\NormalTok{(}\StringTok{\textquotesingle{}Years v/s Movie Duration\textquotesingle{}}\NormalTok{)) }\SpecialCharTok{+} 
        \FunctionTok{ylab}\NormalTok{(}\StringTok{\textquotesingle{}Duration\textquotesingle{}}\NormalTok{) }\SpecialCharTok{+}
          \FunctionTok{xlab}\NormalTok{(}\StringTok{\textquotesingle{}Year\textquotesingle{}}\NormalTok{) }\SpecialCharTok{+}
            \FunctionTok{coord\_flip}\NormalTok{() }\SpecialCharTok{+}
              \FunctionTok{scale\_x\_continuous}\NormalTok{(}\AttributeTok{breaks=}\FunctionTok{seq}\NormalTok{(}\DecValTok{1900}\NormalTok{, }\DecValTok{2020}\NormalTok{, }\DecValTok{10}\NormalTok{))}
\end{Highlighting}
\end{Shaded}

\includegraphics{Midterm_final_word_files/figure-beamer/EDA_year_duration-1.pdf}
\end{block}
\end{frame}

\begin{frame}{Tests Introduction}
\protect\hypertarget{tests-introduction}{}
We have used the following tests to arrive at our conclusions for the
various SMART questions:

\begin{block}{Correlation Test}
\protect\hypertarget{correlation-test}{}
Correlation coefficients are indicators of the strength of the linear
relationship between two different variables: x and y. A linear
correlation coefficient that is greater than zero indicates a positive
relationship. A value that is less than zero signifies a negative
relationship. Finally, a value of zero indicates no relationship between
the two variables x and y. The possible range of values for the
correlation coefficient is -1 to 1. A correlation of -1.0 indicates a
perfect negative correlation, and a correlation of 1.0 indicates a
perfect positive correlation. There are three major types of
correlation: Pearson's Product Moment Co-efficient of Correlation
Spearman's Rank Correlation Coefficient Kendall's Rank Correlation
Coefficient For the Pearson correlation, both variables should be
normally distributed (normally distributed variables have a bell-shaped
curve). Other assumptions include linearity and homoscedasticity.
Linearity assumes a straight-line relationship between each of the two
variables and homoscedasticity assumes that data is equally distributed
about the regression line. The assumptions of the Spearman correlation
are that data must be at least ordinal and the scores on one variable
must be monotonically related to the other variable.
\end{block}

\begin{block}{T -- test}
\protect\hypertarget{t-test}{}
A t-test is a type of inferential statistic used to determine if there
is a significant difference between the means of two groups, which may
be related in certain features. The t-test is one of many tests used for
the purpose of hypothesis testing in statistics. Calculating a t-test
requires three key data values. They include the difference between the
mean values from each data set (called the mean difference), the
standard deviation of each group, and the number of data values of each
group. There are several different types of t-tests that can be
performed depending on the data and type of analysis required.
Mathematically, the t-test takes a sample from each of the two sets and
establishes the problem statement by assuming a null hypothesis that the
two means are equal. Based on the applicable formulas, certain values
are calculated and compared against the standard values, and the assumed
null hypothesis is accepted or rejected accordingly. The pairwise t-test
consists of calculating multiple t-test between all possible
combinations of groups. The Bonferroni correction is a
multiple-comparison correction used when several dependent or
independent statistical tests are being performed simultaneously since
while a given significance level (alpha value) may be appropriate for
each individual comparison, it is not for the set of all comparisons. To
avoid a lot of spurious positives, the alpha value needs to be lowered
to account for the number of comparisons being performed.
\end{block}

\begin{block}{Analysis of Variance (ANOVA)}
\protect\hypertarget{analysis-of-variance-anova}{}
Analysis of variance, or ANOVA, is a statistical method that separates
observed variance data into different components to use for additional
tests. A one-way ANOVA is used for three or more groups of data, to gain
information about the relationship between the dependent and independent
variables. The result of the ANOVA formula, the F statistic (also called
the F-ratio), allows for the analysis of multiple groups of data to
determine the variability between samples and within samples. If no true
variance exists between the groups, the ANOVA's F-ratio should equal
close to 1. There are two main types of ANOVA: one-way (or
unidirectional) and two-way. There are also variations of ANOVA. For
example, MANOVA (multivariate ANOVA) differs from ANOVA as the former
tests for multiple dependent variables simultaneously while the latter
assesses only one dependent variable at a time. One-way or two-way
refers to the number of independent variables in your analysis of
variance test. A one-way ANOVA evaluates the impact of a sole factor on
a sole response variable. It determines whether all the samples are the
same. The one-way ANOVA is used to determine whether there are any
statistically significant differences between the means of three or more
independent (unrelated) groups. ANOVA has many applications in finance,
economics, science, medicine, and social science.
\end{block}

\begin{block}{Tukey's Honest Significant Difference test}
\protect\hypertarget{tukeys-honest-significant-difference-test}{}
The Tukey Test (or Tukey procedure), also called Tukey's Honest
Significant Difference test, is a post-hoc test based on the studentized
range distribution. An ANOVA test can tell you if your results are
significant overall, but it won't tell you exactly where those
differences lie. After you have run an ANOVA and found significant
results, then you can run Tukey's HSD to find out which specific groups'
means (compared with each other) are different. The test compares all
possible pairs of means. If we have unequal sample sizes, we must
calculate the estimated standard deviation for each pairwise comparison.
This is called the Tukey-Kramer Method.
\end{block}
\end{frame}

\begin{frame}[fragile]{SMART Questions}
\protect\hypertarget{smart-questions}{}
There are 21 features in this dataset that include movie genre,
director, released year, actors, duration, movie language, rating, vote
numbers, vote demographics, etc. This project aims to explore the
factors that affect movie ratings and the influence of different ages on
the choice of movie genres. We consider six independent variables: movie
genre, the choice of director, the mean age of actors, the movie
language, the voting demographic (gender and age). These variables
represent different aspects of the influence on movie ratings. Four of
them (movie genre, director, actor age, and the movie language) focus on
the characteristics of the movie. Other variables focus on the
descriptions of voters, like their age and gender.

We conducted exploratory data analysis by selecting quantitative
variables like movie rating, plotting histograms, and bar graphs. During
the EDA, one quantitative variable, movie duration, seemed to could be
taken into account. We resolved to analyze the relationship between
movie duration and ratings and rejected one question regarding the actor
age because of increased complexity.

\begin{block}{Difference in Movie Ratings Across Genres}
\protect\hypertarget{difference-in-movie-ratings-across-genres}{}
We split up the movie genre column into its sub-genres and assigned the
same rating to each sub-genre. Outliers in the avg\_vote column were
removed since we need the votes to follow a normal distribution for our
upcoming hypothesis tests. Grouping the data to find the mean rating for
every genre returned the following results:

\begin{Shaded}
\begin{Highlighting}[]
\NormalTok{genre\_df1 }\OtherTok{\textless{}{-}}\NormalTok{ movie\_ratings[,}\FunctionTok{c}\NormalTok{(}\StringTok{"genre"}\NormalTok{, }\StringTok{"avg\_vote"}\NormalTok{)]}

\CommentTok{\# remove outliers}
\NormalTok{outliers }\OtherTok{\textless{}{-}} \FunctionTok{unique}\NormalTok{(}\FunctionTok{boxplot}\NormalTok{(genre\_df1}\SpecialCharTok{$}\NormalTok{avg\_vote, }\AttributeTok{plot=}\ConstantTok{FALSE}\NormalTok{)}\SpecialCharTok{$}\NormalTok{out)}
\NormalTok{genre\_df1 }\OtherTok{\textless{}{-}}\NormalTok{ genre\_df1[}\SpecialCharTok{{-}}\FunctionTok{which}\NormalTok{(genre\_df1}\SpecialCharTok{$}\NormalTok{avg\_vote }\SpecialCharTok{\%in\%}\NormalTok{ outliers),]}

\NormalTok{genre\_df2 }\OtherTok{\textless{}{-}}\NormalTok{ genre\_df1 }\SpecialCharTok{\%\textgreater{}\%}
  \FunctionTok{separate}\NormalTok{(genre, }\FunctionTok{c}\NormalTok{(}\StringTok{"c1"}\NormalTok{,}\StringTok{\textquotesingle{}c2\textquotesingle{}}\NormalTok{,}\StringTok{\textquotesingle{}c3\textquotesingle{}}\NormalTok{),}\AttributeTok{sep =}\FunctionTok{c}\NormalTok{(}\StringTok{\textquotesingle{}, \textquotesingle{}}\NormalTok{))}

\NormalTok{df1 }\OtherTok{\textless{}{-}}\NormalTok{ genre\_df2[,}\FunctionTok{c}\NormalTok{(}\StringTok{"c1"}\NormalTok{, }\StringTok{"avg\_vote"}\NormalTok{)] }\SpecialCharTok{\%\textgreater{}\%} \FunctionTok{drop\_na}\NormalTok{()}
\NormalTok{df2 }\OtherTok{\textless{}{-}}\NormalTok{ genre\_df2[,}\FunctionTok{c}\NormalTok{(}\StringTok{"c2"}\NormalTok{, }\StringTok{"avg\_vote"}\NormalTok{)] }\SpecialCharTok{\%\textgreater{}\%} \FunctionTok{drop\_na}\NormalTok{()}
\NormalTok{df3 }\OtherTok{\textless{}{-}}\NormalTok{ genre\_df2[,}\FunctionTok{c}\NormalTok{(}\StringTok{"c3"}\NormalTok{, }\StringTok{"avg\_vote"}\NormalTok{)] }\SpecialCharTok{\%\textgreater{}\%} \FunctionTok{drop\_na}\NormalTok{()}

\FunctionTok{names}\NormalTok{(df1)[}\DecValTok{1}\NormalTok{] }\OtherTok{\textless{}{-}} \StringTok{"genre"}
\FunctionTok{names}\NormalTok{(df2)[}\DecValTok{1}\NormalTok{] }\OtherTok{\textless{}{-}} \StringTok{"genre"}
\FunctionTok{names}\NormalTok{(df3)[}\DecValTok{1}\NormalTok{] }\OtherTok{\textless{}{-}} \StringTok{"genre"}

\NormalTok{genre\_df }\OtherTok{\textless{}{-}} \FunctionTok{rbind}\NormalTok{(df1, df2, df3)}
\end{Highlighting}
\end{Shaded}

\begin{block}{Data Visualization}
\protect\hypertarget{data-visualization}{}
\begin{Shaded}
\begin{Highlighting}[]
\FunctionTok{theme\_set}\NormalTok{(}\FunctionTok{theme\_bw}\NormalTok{())}

\NormalTok{genre\_grouped }\OtherTok{\textless{}{-}}\NormalTok{ genre\_df }\SpecialCharTok{\%\textgreater{}\%}
  \FunctionTok{group\_by}\NormalTok{(genre) }\SpecialCharTok{\%\textgreater{}\%}
  \FunctionTok{filter}\NormalTok{(}\FunctionTok{n}\NormalTok{() }\SpecialCharTok{\textgreater{}=} \DecValTok{30}\NormalTok{) }\SpecialCharTok{\%\textgreater{}\%}
    \FunctionTok{summarize}\NormalTok{(}
      \AttributeTok{mean\_vote =} \FunctionTok{mean}\NormalTok{(avg\_vote),}
      \AttributeTok{n=}\FunctionTok{n}\NormalTok{()}
\NormalTok{    )}

\NormalTok{genre\_grouped }\OtherTok{\textless{}{-}}\NormalTok{ genre\_grouped[}\FunctionTok{order}\NormalTok{(genre\_grouped}\SpecialCharTok{$}\NormalTok{mean\_vote), ]}
\NormalTok{genre\_grouped}\SpecialCharTok{$}\NormalTok{genre }\OtherTok{\textless{}{-}} \FunctionTok{factor}\NormalTok{(genre\_grouped}\SpecialCharTok{$}\NormalTok{genre, }\AttributeTok{levels =}\NormalTok{ genre\_grouped}\SpecialCharTok{$}\NormalTok{genre)}

\FunctionTok{ggplot}\NormalTok{(genre\_grouped, }\FunctionTok{aes}\NormalTok{(}\AttributeTok{fill=}\NormalTok{mean\_vote, }\AttributeTok{y=}\NormalTok{mean\_vote, }\AttributeTok{x=}\NormalTok{genre)) }\SpecialCharTok{+}
    \FunctionTok{geom\_bar}\NormalTok{(}\AttributeTok{stat=}\StringTok{"identity"}\NormalTok{, }\AttributeTok{alpha=}\NormalTok{.}\DecValTok{8}\NormalTok{, }\AttributeTok{width=}\FloatTok{0.8}\NormalTok{) }\SpecialCharTok{+}
      \FunctionTok{scale\_fill\_fermenter}\NormalTok{(}\AttributeTok{palette =} \StringTok{"Set2"}\NormalTok{) }\SpecialCharTok{+}
        \FunctionTok{coord\_flip}\NormalTok{() }\SpecialCharTok{+}
            \FunctionTok{labs}\NormalTok{(}\AttributeTok{title=}\StringTok{"Ordered Scatter Plot"}\NormalTok{,}
                \AttributeTok{subtitle=}\StringTok{"Mean avg\_vote across film genres"}\NormalTok{, }
                \AttributeTok{caption=}\StringTok{"genre: avg\_vote"}\NormalTok{)}
\end{Highlighting}
\end{Shaded}

\includegraphics{Midterm_final_word_files/figure-beamer/Q1_visualization-1.pdf}
\end{block}

\begin{block}{Hypothesis Testing and P-value visualization}
\protect\hypertarget{hypothesis-testing-and-p-value-visualization}{}
We can see from the visualization that there is a difference among some
genres based on the mean ratings they receive. We can run a pairwise
t-test on the dataset to find out which pairs have a significant
difference in mean ratings. We set up our experiment with an alpha-value
of 0.01 as follows:

Ho -\textgreater{} The mean rating for all movie genres is same

Ha -\textgreater{} The mean rating for all movie genres is not same

Due to there being 25 different groups of genres, we have also elected
to use the Bonferroni correction to get adjusted p-values. The results
of our test can be best portrayed with the help of a heatmap of p-values
for difference in mean ratings between the various genres. As we can see
from the map, most of the values are very close to zero, which means
that the genre of the movie has a significant effect on the ratings the
movie is going to receive.

\begin{Shaded}
\begin{Highlighting}[]
\NormalTok{genre\_df\_filtered }\OtherTok{\textless{}{-}}\NormalTok{ genre\_df }\SpecialCharTok{\%\textgreater{}\%}
  \FunctionTok{group\_by}\NormalTok{(genre) }\SpecialCharTok{\%\textgreater{}\%}
    \FunctionTok{filter}\NormalTok{(}\FunctionTok{n}\NormalTok{() }\SpecialCharTok{\textgreater{}=} \DecValTok{30}\NormalTok{) }\SpecialCharTok{\%\textgreater{}\%}
      \FunctionTok{ungroup}\NormalTok{()}

\NormalTok{pw\_ttest }\OtherTok{\textless{}{-}} \FunctionTok{pairwise.t.test}\NormalTok{(}
  \AttributeTok{x=}\NormalTok{genre\_df\_filtered}\SpecialCharTok{$}\NormalTok{avg\_vote, }
  \AttributeTok{g=}\NormalTok{genre\_df\_filtered}\SpecialCharTok{$}\NormalTok{genre,}
  \AttributeTok{p.adjust.method=}\StringTok{"bonferroni"}
\NormalTok{)}

\NormalTok{pw\_ttest\_pvals }\OtherTok{\textless{}{-}} \FunctionTok{data.frame}\NormalTok{(pw\_ttest}\SpecialCharTok{$}\NormalTok{p.value)}
\NormalTok{pw\_ttest\_pvals}\SpecialCharTok{$}\NormalTok{genre1 }\OtherTok{\textless{}{-}} \FunctionTok{rownames}\NormalTok{(pw\_ttest\_pvals)}
\NormalTok{pw\_ttest\_pvals }\OtherTok{\textless{}{-}} \FunctionTok{melt}\NormalTok{(pw\_ttest\_pvals, }\AttributeTok{id.vars=}\StringTok{"genre1"}\NormalTok{, }\AttributeTok{variable.name=}\StringTok{"genre2"}\NormalTok{, }\AttributeTok{value.name=}\StringTok{"p\_value"}\NormalTok{) }\SpecialCharTok{\%\textgreater{}\%} \FunctionTok{drop\_na}\NormalTok{()}

\FunctionTok{ggplot}\NormalTok{(pw\_ttest\_pvals, }\FunctionTok{aes}\NormalTok{(genre1, genre2, }\AttributeTok{fill=}\NormalTok{ p\_value)) }\SpecialCharTok{+} 
  \FunctionTok{geom\_tile}\NormalTok{() }\SpecialCharTok{+}
    \FunctionTok{scale\_fill\_gradient}\NormalTok{(}\AttributeTok{low=}\StringTok{"darkblue"}\NormalTok{, }\AttributeTok{high =} \StringTok{"lightblue"}\NormalTok{) }\SpecialCharTok{+}
      \FunctionTok{theme}\NormalTok{(}\AttributeTok{axis.text.x =} \FunctionTok{element\_text}\NormalTok{(}\AttributeTok{angle=}\DecValTok{45}\NormalTok{, }\AttributeTok{vjust=}\DecValTok{1}\NormalTok{, }\AttributeTok{hjust=}\DecValTok{1}\NormalTok{)) }\SpecialCharTok{+}
        \FunctionTok{labs}\NormalTok{(}\AttributeTok{title=}\StringTok{"Heat Map"}\NormalTok{,}
            \AttributeTok{subtitle=}\StringTok{"P{-}values for pairwise T{-}tests on movie genres"}\NormalTok{, }
            \AttributeTok{caption=}\StringTok{"genre2: genre1"}\NormalTok{)}
\end{Highlighting}
\end{Shaded}

\includegraphics{Midterm_final_word_files/figure-beamer/Q1_hypo_test-1.pdf}
\end{block}
\end{block}

\begin{block}{Choice of Director affect movie ratings}
\protect\hypertarget{choice-of-director-affect-movie-ratings}{}
We group up the movie data on the director and find the mean rating that
each director has received for their movies. We only consider directors
who have created at least 50 movies for visualization purposes, and we
get the following results:

\begin{Shaded}
\begin{Highlighting}[]
\NormalTok{movie\_director }\OtherTok{\textless{}{-}}\NormalTok{ movie\_ratings[,}\FunctionTok{c}\NormalTok{(}\StringTok{"director"}\NormalTok{, }\StringTok{"avg\_vote"}\NormalTok{)]}

\CommentTok{\# remove outliers}
\NormalTok{outliers }\OtherTok{\textless{}{-}} \FunctionTok{unique}\NormalTok{(}\FunctionTok{boxplot}\NormalTok{(movie\_director}\SpecialCharTok{$}\NormalTok{avg\_vote, }\AttributeTok{plot=}\ConstantTok{FALSE}\NormalTok{)}\SpecialCharTok{$}\NormalTok{out)}
\NormalTok{movie\_director }\OtherTok{\textless{}{-}}\NormalTok{ movie\_director[}\SpecialCharTok{{-}}\FunctionTok{which}\NormalTok{(movie\_director}\SpecialCharTok{$}\NormalTok{avg\_vote }\SpecialCharTok{\%in\%}\NormalTok{ outliers),]}

\NormalTok{director\_group }\OtherTok{\textless{}{-}} \FunctionTok{aggregate}\NormalTok{(movie\_director}\SpecialCharTok{$}\NormalTok{avg\_vote, }\AttributeTok{by =} \FunctionTok{list}\NormalTok{(}\AttributeTok{director =}\NormalTok{ movie\_director}\SpecialCharTok{$}\NormalTok{director),}\AttributeTok{FUN=}\NormalTok{mean)}
\NormalTok{director\_num }\OtherTok{\textless{}{-}}\NormalTok{ movie\_director }\SpecialCharTok{\%\textgreater{}\%} \FunctionTok{count}\NormalTok{(director,}\AttributeTok{sort =} \ConstantTok{TRUE}\NormalTok{)}
\NormalTok{director\_vote }\OtherTok{\textless{}{-}} \FunctionTok{merge}\NormalTok{(director\_group,director\_num,}\AttributeTok{by=}\StringTok{\textquotesingle{}director\textquotesingle{}}\NormalTok{)}
\NormalTok{director\_vote }\OtherTok{\textless{}{-}}\NormalTok{ director\_vote[}\FunctionTok{order}\NormalTok{(}\SpecialCharTok{{-}}\NormalTok{director\_vote}\SpecialCharTok{$}\NormalTok{n),]}
\FunctionTok{colnames}\NormalTok{(director\_vote) }\OtherTok{\textless{}{-}} \FunctionTok{c}\NormalTok{(}\StringTok{\textquotesingle{}director\textquotesingle{}}\NormalTok{,}\StringTok{\textquotesingle{}rating\textquotesingle{}}\NormalTok{,}\StringTok{\textquotesingle{}movie\_numbers\textquotesingle{}}\NormalTok{) }

\NormalTok{director\_vote\_50 }\OtherTok{\textless{}{-}} \FunctionTok{subset}\NormalTok{(director\_vote,movie\_numbers}\SpecialCharTok{\textgreater{}}\DecValTok{49}\NormalTok{)}
\NormalTok{director\_vote\_50}\SpecialCharTok{$}\NormalTok{rating }\OtherTok{\textless{}{-}} \FunctionTok{as.numeric}\NormalTok{(director\_vote\_50}\SpecialCharTok{$}\NormalTok{rating)}

\FunctionTok{ggplot}\NormalTok{(director\_vote\_50, }\FunctionTok{aes}\NormalTok{(}\AttributeTok{fill=}\NormalTok{rating, }\AttributeTok{y=}\NormalTok{rating, }\AttributeTok{x=}\FunctionTok{reorder}\NormalTok{(director,rating))) }\SpecialCharTok{+}
    \FunctionTok{geom\_bar}\NormalTok{(}\AttributeTok{stat=}\StringTok{"identity"}\NormalTok{, }\AttributeTok{alpha=}\NormalTok{.}\DecValTok{8}\NormalTok{, }\AttributeTok{width=}\FloatTok{0.8}\NormalTok{) }\SpecialCharTok{+}
      \FunctionTok{scale\_fill\_fermenter}\NormalTok{(}\AttributeTok{palette =} \StringTok{"Set2"}\NormalTok{) }\SpecialCharTok{+}
        \FunctionTok{coord\_flip}\NormalTok{() }\SpecialCharTok{+}
            \FunctionTok{labs}\NormalTok{(}\AttributeTok{title=}\StringTok{"Ordered Bar Chart"}\NormalTok{,}
                \AttributeTok{subtitle=}\StringTok{"Mean avg\_vote for various Directors"}\NormalTok{,}
                \AttributeTok{caption=}\StringTok{"director: avg\_vote"}\NormalTok{)}
\end{Highlighting}
\end{Shaded}

\includegraphics{Midterm_final_word_files/figure-beamer/Q2-1.pdf}

\begin{block}{Hypothesis Testing and P-value visualization}
\protect\hypertarget{hypothesis-testing-and-p-value-visualization-1}{}
At first glance, we can see that there is a considerable difference in
movie ratings for different directors. To back up our findings, we run
an ANOVA test on the data with a significance level of 0.01 as follows:

Ho -\textgreater{} The mean movie ratings for all directors is same

Ha -\textgreater{} The mean movie ratings for all directors is not same

The results of the ANOVA test are:

\begin{Shaded}
\begin{Highlighting}[]
\NormalTok{movie\_director\_filtered }\OtherTok{\textless{}{-}}\NormalTok{ movie\_director }\SpecialCharTok{\%\textgreater{}\%}
  \FunctionTok{group\_by}\NormalTok{(director) }\SpecialCharTok{\%\textgreater{}\%}
    \FunctionTok{filter}\NormalTok{(}\FunctionTok{n}\NormalTok{() }\SpecialCharTok{\textgreater{}=} \DecValTok{30}\NormalTok{) }\SpecialCharTok{\%\textgreater{}\%}
      \FunctionTok{ungroup}\NormalTok{()}

\NormalTok{anovaRes }\OtherTok{=} \FunctionTok{aov}\NormalTok{(avg\_vote }\SpecialCharTok{\textasciitilde{}}\NormalTok{ director, }\AttributeTok{data=}\NormalTok{movie\_director\_filtered)}
\NormalTok{anovaRes }
\end{Highlighting}
\end{Shaded}

\begin{verbatim}
## Call:
##    aov(formula = avg_vote ~ director, data = movie_director_filtered)
## 
## Terms:
##                 director Residuals
## Sum of Squares      1260      1726
## Deg. of Freedom       91      3537
## 
## Residual standard error: 0.698
## Estimated effects may be unbalanced
\end{verbatim}

\begin{Shaded}
\begin{Highlighting}[]
\FunctionTok{summary}\NormalTok{(anovaRes)}
\end{Highlighting}
\end{Shaded}

\begin{verbatim}
##               Df Sum Sq Mean Sq F value Pr(>F)    
## director      91   1260   13.85    28.4 <2e-16 ***
## Residuals   3537   1726    0.49                   
## ---
## Signif. codes:  0 '***' 0.001 '**' 0.01 '*' 0.05 '.' 0.1 ' ' 1
\end{verbatim}

Running a post-hoc Tukey-Kramer test on the ANOVA results with a 99\%
confidence interval give us the director pairs which significantly
differ in the mean of the ratings of their movies. A snippet of the
output is as follows:

\begin{Shaded}
\begin{Highlighting}[]
\NormalTok{tukeyDirectorAoV }\OtherTok{\textless{}{-}} \FunctionTok{TukeyHSD}\NormalTok{(anovaRes)}
\NormalTok{tukeyVar }\OtherTok{\textless{}{-}} \FunctionTok{data.frame}\NormalTok{(tukeyDirectorAoV}\SpecialCharTok{$}\NormalTok{director)}
\NormalTok{tukeyVarSig }\OtherTok{\textless{}{-}} \FunctionTok{subset}\NormalTok{(tukeyVar, p.adj }\SpecialCharTok{\textless{}=} \FloatTok{0.01}\NormalTok{)}
\FunctionTok{head}\NormalTok{(tukeyVarSig)}
\end{Highlighting}
\end{Shaded}

\begin{verbatim}
##                                         diff   lwr    upr    p.adj
## Carlo Vanzina-Alekos Sakellarios      -2.023 -2.69 -1.354 3.77e-09
## David Dhawan-Alekos Sakellarios       -1.594 -2.30 -0.890 3.77e-09
## Fred Olen Ray-Alekos Sakellarios      -3.068 -3.77 -2.364 3.77e-09
## George Sherman-Alekos Sakellarios     -0.834 -1.54 -0.125 1.85e-03
## Gerald Thomas-Alekos Sakellarios      -0.997 -1.70 -0.298 4.91e-06
## Giannis Dalianidis-Alekos Sakellarios -0.971 -1.61 -0.330 4.81e-07
\end{verbatim}

Our final director dataset has 92 directors who had more than or equal
to 30 films to their name. This is done to ensure that groups with
lesser number of samples do not distort the test findings. As a result
of this, we could get a maximum of 4186 pairs of directors with
significant differences in their movie ratings. The Tukey test return a
table of 840 pairs of directors with significant movie rating
differences. Hence, we can reasonably say that the choice of director is
important in selecting a movie.
\end{block}
\end{block}

\begin{block}{Movie genre viewing statistic differences with age}
\protect\hypertarget{movie-genre-viewing-statistic-differences-with-age}{}
To answer this SMART question, we firstly take the proportion of viewer
votes for different age ranges. We have 4 age ranges in our dataset: 0
-- 18 age range 18 -- 30 age range 30 -- 45 age range 45+ age range

The 0-18 age range has many missing values, so we do not consider those
votes. For the rest of the values, we name the columns as ages\_18,
ages\_30, and ages\_45 respectively. We also name their relative
proportions as prop\_18, prop\_30, and prop\_45 respectively. We see the
following distributions of densities across different movie genres for
the various age groups:

\begin{Shaded}
\begin{Highlighting}[]
\NormalTok{genre\_age\_df1 }\OtherTok{\textless{}{-}}\NormalTok{ movie\_ratings[,}\FunctionTok{c}\NormalTok{(}\StringTok{"genre"}\NormalTok{, }\StringTok{"allgenders\_18age\_votes"}\NormalTok{, }\StringTok{"allgenders\_30age\_votes"}\NormalTok{, }\StringTok{"allgenders\_45age\_votes"}\NormalTok{)]}

\NormalTok{genre\_age\_df2 }\OtherTok{\textless{}{-}}\NormalTok{ genre\_age\_df1 }\SpecialCharTok{\%\textgreater{}\%}
    \FunctionTok{mutate}\NormalTok{(}
      \AttributeTok{prop\_18 =}\NormalTok{ allgenders\_18age\_votes }\SpecialCharTok{/}\NormalTok{ (allgenders\_18age\_votes }\SpecialCharTok{+}\NormalTok{ allgenders\_30age\_votes }\SpecialCharTok{+}\NormalTok{ allgenders\_45age\_votes),}
      \AttributeTok{prop\_30 =}\NormalTok{ allgenders\_30age\_votes }\SpecialCharTok{/}\NormalTok{ (allgenders\_18age\_votes }\SpecialCharTok{+}\NormalTok{ allgenders\_30age\_votes }\SpecialCharTok{+}\NormalTok{ allgenders\_45age\_votes),}
      \AttributeTok{prop\_45 =}\NormalTok{ allgenders\_45age\_votes }\SpecialCharTok{/}\NormalTok{ (allgenders\_18age\_votes }\SpecialCharTok{+}\NormalTok{ allgenders\_30age\_votes }\SpecialCharTok{+}\NormalTok{ allgenders\_45age\_votes)}
\NormalTok{    )}

\FunctionTok{names}\NormalTok{(genre\_age\_df2)[}\DecValTok{2}\SpecialCharTok{:}\DecValTok{4}\NormalTok{] }\OtherTok{\textless{}{-}} \FunctionTok{c}\NormalTok{(}\StringTok{"votes\_18"}\NormalTok{, }\StringTok{"votes\_30"}\NormalTok{, }\StringTok{"votes\_45"}\NormalTok{)}

\NormalTok{genre\_age\_df3 }\OtherTok{\textless{}{-}}\NormalTok{ genre\_age\_df2 }\SpecialCharTok{\%\textgreater{}\%}
  \FunctionTok{separate}\NormalTok{(genre, }\FunctionTok{c}\NormalTok{(}\StringTok{"c1"}\NormalTok{,}\StringTok{\textquotesingle{}c2\textquotesingle{}}\NormalTok{,}\StringTok{\textquotesingle{}c3\textquotesingle{}}\NormalTok{),}\AttributeTok{sep =}\FunctionTok{c}\NormalTok{(}\StringTok{\textquotesingle{}, \textquotesingle{}}\NormalTok{))}

\NormalTok{df1 }\OtherTok{\textless{}{-}}\NormalTok{ genre\_age\_df3[,}\FunctionTok{c}\NormalTok{(}\StringTok{"c1"}\NormalTok{, }\StringTok{"votes\_18"}\NormalTok{, }\StringTok{"votes\_30"}\NormalTok{, }\StringTok{"votes\_45"}\NormalTok{, }\StringTok{"prop\_18"}\NormalTok{, }\StringTok{"prop\_30"}\NormalTok{, }\StringTok{"prop\_45"}\NormalTok{)] }\SpecialCharTok{\%\textgreater{}\%} \FunctionTok{drop\_na}\NormalTok{()}
\NormalTok{df2 }\OtherTok{\textless{}{-}}\NormalTok{ genre\_age\_df3[,}\FunctionTok{c}\NormalTok{(}\StringTok{"c2"}\NormalTok{, }\StringTok{"votes\_18"}\NormalTok{, }\StringTok{"votes\_30"}\NormalTok{, }\StringTok{"votes\_45"}\NormalTok{, }\StringTok{"prop\_18"}\NormalTok{, }\StringTok{"prop\_30"}\NormalTok{, }\StringTok{"prop\_45"}\NormalTok{)] }\SpecialCharTok{\%\textgreater{}\%} \FunctionTok{drop\_na}\NormalTok{()}
\NormalTok{df3 }\OtherTok{\textless{}{-}}\NormalTok{ genre\_age\_df3[,}\FunctionTok{c}\NormalTok{(}\StringTok{"c3"}\NormalTok{, }\StringTok{"votes\_18"}\NormalTok{, }\StringTok{"votes\_30"}\NormalTok{, }\StringTok{"votes\_45"}\NormalTok{, }\StringTok{"prop\_18"}\NormalTok{, }\StringTok{"prop\_30"}\NormalTok{, }\StringTok{"prop\_45"}\NormalTok{)] }\SpecialCharTok{\%\textgreater{}\%} \FunctionTok{drop\_na}\NormalTok{()}

\FunctionTok{names}\NormalTok{(df1)[}\DecValTok{1}\NormalTok{] }\OtherTok{\textless{}{-}} \StringTok{"genre"}
\FunctionTok{names}\NormalTok{(df2)[}\DecValTok{1}\NormalTok{] }\OtherTok{\textless{}{-}} \StringTok{"genre"}
\FunctionTok{names}\NormalTok{(df3)[}\DecValTok{1}\NormalTok{] }\OtherTok{\textless{}{-}} \StringTok{"genre"}

\NormalTok{genre\_age\_df }\OtherTok{\textless{}{-}} \FunctionTok{rbind}\NormalTok{(df1, df2, df3)}
\end{Highlighting}
\end{Shaded}

\begin{block}{Data Visualization}
\protect\hypertarget{data-visualization-1}{}
Visualizing the data in a bar chart format to better understand the
proportions for various genres gives us the following graph:

\begin{Shaded}
\begin{Highlighting}[]
\FunctionTok{ggplot}\NormalTok{(genre\_age\_df, }\FunctionTok{aes}\NormalTok{(}\AttributeTok{x=}\NormalTok{prop\_18, }\AttributeTok{group=}\NormalTok{genre, }\AttributeTok{fill=}\NormalTok{genre)) }\SpecialCharTok{+}
  \FunctionTok{geom\_density}\NormalTok{(}\AttributeTok{adjust=}\FloatTok{1.5}\NormalTok{, }\AttributeTok{alpha=}\NormalTok{.}\DecValTok{4}\NormalTok{) }\SpecialCharTok{+}
      \FunctionTok{ggtitle}\NormalTok{(}\StringTok{"Distribution of Age 18 viewing proportions for various genres"}\NormalTok{)}
\end{Highlighting}
\end{Shaded}

\includegraphics{Midterm_final_word_files/figure-beamer/Q3_visualization-1.pdf}

\begin{Shaded}
\begin{Highlighting}[]
\FunctionTok{ggplot}\NormalTok{(genre\_age\_df, }\FunctionTok{aes}\NormalTok{(}\AttributeTok{x=}\NormalTok{prop\_30, }\AttributeTok{group=}\NormalTok{genre, }\AttributeTok{fill=}\NormalTok{genre)) }\SpecialCharTok{+}
  \FunctionTok{geom\_density}\NormalTok{(}\AttributeTok{adjust=}\FloatTok{1.5}\NormalTok{, }\AttributeTok{alpha=}\NormalTok{.}\DecValTok{4}\NormalTok{) }\SpecialCharTok{+}
      \FunctionTok{ggtitle}\NormalTok{(}\StringTok{"Distribution of Age 30 viewing proportions for various genres"}\NormalTok{)}
\end{Highlighting}
\end{Shaded}

\includegraphics{Midterm_final_word_files/figure-beamer/Q3_visualization-2.pdf}

\begin{Shaded}
\begin{Highlighting}[]
\FunctionTok{ggplot}\NormalTok{(genre\_age\_df, }\FunctionTok{aes}\NormalTok{(}\AttributeTok{x=}\NormalTok{prop\_45, }\AttributeTok{group=}\NormalTok{genre, }\AttributeTok{fill=}\NormalTok{genre)) }\SpecialCharTok{+}
  \FunctionTok{geom\_density}\NormalTok{(}\AttributeTok{adjust=}\FloatTok{1.5}\NormalTok{, }\AttributeTok{alpha=}\NormalTok{.}\DecValTok{4}\NormalTok{) }\SpecialCharTok{+}
      \FunctionTok{ggtitle}\NormalTok{(}\StringTok{"Distribution of Age 45 viewing proportions for various genres"}\NormalTok{)}
\end{Highlighting}
\end{Shaded}

\includegraphics{Midterm_final_word_files/figure-beamer/Q3_visualization-3.pdf}

\begin{Shaded}
\begin{Highlighting}[]
\NormalTok{genre\_age\_grouped }\OtherTok{\textless{}{-}}\NormalTok{ genre\_age\_df }\SpecialCharTok{\%\textgreater{}\%}
  \FunctionTok{group\_by}\NormalTok{(genre) }\SpecialCharTok{\%\textgreater{}\%}
    \FunctionTok{summarize}\NormalTok{(}
      \AttributeTok{ages\_18 =} \FunctionTok{mean}\NormalTok{(prop\_18),}
      \AttributeTok{ages\_30 =} \FunctionTok{mean}\NormalTok{(prop\_30),}
      \AttributeTok{ages\_45 =} \FunctionTok{mean}\NormalTok{(prop\_45)}
\NormalTok{    )}

\NormalTok{genre\_age\_melt }\OtherTok{\textless{}{-}} \FunctionTok{melt}\NormalTok{(genre\_age\_grouped, }\AttributeTok{id.vars =} \StringTok{"genre"}\NormalTok{, }\AttributeTok{variable.name =} \StringTok{"proportions"}\NormalTok{)}

\FunctionTok{ggplot}\NormalTok{(genre\_age\_melt, }\FunctionTok{aes}\NormalTok{(}\AttributeTok{fill=}\NormalTok{proportions, }\AttributeTok{y=}\NormalTok{value, }\AttributeTok{x=}\NormalTok{genre)) }\SpecialCharTok{+} 
    \FunctionTok{geom\_bar}\NormalTok{(}\AttributeTok{position=}\StringTok{"stack"}\NormalTok{, }\AttributeTok{stat=}\StringTok{"identity"}\NormalTok{) }\SpecialCharTok{+}
        \FunctionTok{scale\_fill\_brewer}\NormalTok{(}\AttributeTok{palette =} \StringTok{"Set2"}\NormalTok{) }\SpecialCharTok{+}
            \FunctionTok{coord\_flip}\NormalTok{() }\SpecialCharTok{+} 
                \FunctionTok{ggtitle}\NormalTok{(}\StringTok{"Mean of proportions of viewer ages across film genres"}\NormalTok{)}
\end{Highlighting}
\end{Shaded}

\includegraphics{Midterm_final_word_files/figure-beamer/Q3_visualization-4.pdf}
\end{block}

\begin{block}{Hypothesis Testing}
\protect\hypertarget{hypothesis-testing}{}
To see if there are significant differences in viewing statistics across
genres for different ages, we run a MANOVA (Multivariate ANOVA) test. We
set up our experiment with significance level 0.01 as follows:

Ho -\textgreater{} The viewing statistic for different ages across
multiple genres is same

Ha -\textgreater{} The viewing statistic for different ages across
multiple genres is not same

The results of the MANOVA test are:

\begin{Shaded}
\begin{Highlighting}[]
\NormalTok{genre\_age\_df\_filtered }\OtherTok{\textless{}{-}}\NormalTok{ genre\_age\_df }\SpecialCharTok{\%\textgreater{}\%}
  \FunctionTok{group\_by}\NormalTok{(genre) }\SpecialCharTok{\%\textgreater{}\%}
    \FunctionTok{filter}\NormalTok{(}\FunctionTok{n}\NormalTok{() }\SpecialCharTok{\textgreater{}=} \DecValTok{30}\NormalTok{) }\SpecialCharTok{\%\textgreater{}\%}
      \FunctionTok{ungroup}\NormalTok{()}

\NormalTok{res.man }\OtherTok{\textless{}{-}} \FunctionTok{manova}\NormalTok{(}\FunctionTok{cbind}\NormalTok{(votes\_18, votes\_30, votes\_45) }\SpecialCharTok{\textasciitilde{}}\NormalTok{ genre, }\AttributeTok{data =}\NormalTok{ genre\_age\_df\_filtered)}
\FunctionTok{summary}\NormalTok{(res.man)}
\end{Highlighting}
\end{Shaded}

\begin{verbatim}
##               Df Pillai approx F num Df den Df Pr(>F)    
## genre         20 0.0235     61.3     60 466449 <2e-16 ***
## Residuals 155483                                         
## ---
## Signif. codes:  0 '***' 0.001 '**' 0.01 '*' 0.05 '.' 0.1 ' ' 1
\end{verbatim}

A very low p-value tells us that genre differences are significant
across various ages. We check which age ranges have significant
differences between various genre movie views as follows:

\begin{Shaded}
\begin{Highlighting}[]
\FunctionTok{summary.aov}\NormalTok{(res.man)}
\end{Highlighting}
\end{Shaded}

\begin{verbatim}
##  Response votes_18 :
##                 Df   Sum Sq  Mean Sq F value Pr(>F)    
## genre           20 3.44e+11 1.72e+10    87.9 <2e-16 ***
## Residuals   155483 3.04e+13 1.96e+08                   
## ---
## Signif. codes:  0 '***' 0.001 '**' 0.01 '*' 0.05 '.' 0.1 ' ' 1
## 
##  Response votes_30 :
##                 Df   Sum Sq  Mean Sq F value Pr(>F)    
## genre           20 1.35e+12 6.76e+10    99.8 <2e-16 ***
## Residuals   155483 1.05e+14 6.78e+08                   
## ---
## Signif. codes:  0 '***' 0.001 '**' 0.01 '*' 0.05 '.' 0.1 ' ' 1
## 
##  Response votes_45 :
##                 Df   Sum Sq  Mean Sq F value Pr(>F)    
## genre           20 8.22e+10 4.11e+09     104 <2e-16 ***
## Residuals   155483 6.16e+12 3.96e+07                   
## ---
## Signif. codes:  0 '***' 0.001 '**' 0.01 '*' 0.05 '.' 0.1 ' ' 1
\end{verbatim}

A low p-value for each age category tells us that all the ages have
differing viewing statistics across different genres.
\end{block}
\end{block}

\begin{block}{Affect of duration of movie on avg\_vote}
\protect\hypertarget{affect-of-duration-of-movie-on-avg_vote}{}
\begin{block}{Outliers Analysis}
\protect\hypertarget{outliers-analysis}{}
To see if the duration of the movie influences the movie rating, we must
first remove outliers from the dataset. Here is a distribution of the
duration values with and without outliers: Plotting the duration of the
movie against the mean rating for the movie gives us the following
results:

\begin{Shaded}
\begin{Highlighting}[]
\CommentTok{\# movies duration{-}rating distribution}
\NormalTok{duration }\OtherTok{\textless{}{-}}\FunctionTok{outlierKD2}\NormalTok{(movies, movies}\SpecialCharTok{$}\NormalTok{duration)}
\end{Highlighting}
\end{Shaded}

\begin{verbatim}
## Outliers identified: 5671 
## Propotion (%) of outliers: 7.2 
## Mean of the outliers: 149 
## Mean without removing outliers: 100 
## Mean if we remove outliers: 96.8 
## Nothing changed
\end{verbatim}

\begin{Shaded}
\begin{Highlighting}[]
\NormalTok{outliers }\OtherTok{\textless{}{-}} \FunctionTok{boxplot}\NormalTok{(movies}\SpecialCharTok{$}\NormalTok{duration,}\AttributeTok{plot=}\ConstantTok{FALSE}\NormalTok{)}\SpecialCharTok{$}\NormalTok{out}
\NormalTok{movies1 }\OtherTok{\textless{}{-}}\NormalTok{ movies[}\SpecialCharTok{{-}}\FunctionTok{which}\NormalTok{(movies}\SpecialCharTok{$}\NormalTok{duration }\SpecialCharTok{\%in\%}\NormalTok{ outliers),]}
\FunctionTok{summary}\NormalTok{(movies1}\SpecialCharTok{$}\NormalTok{duration)}
\end{Highlighting}
\end{Shaded}

\begin{verbatim}
##    Min. 1st Qu.  Median    Mean 3rd Qu.    Max. 
##    58.0    88.0    95.0    96.8   105.0   138.0
\end{verbatim}

\includegraphics{Midterm_final_word_files/figure-beamer/Q4_EDA-1.pdf}

Plotting the duration of the movie against the mean rating for the movie
gives us the following results:

\begin{Shaded}
\begin{Highlighting}[]
\NormalTok{duration\_rating }\OtherTok{\textless{}{-}}\NormalTok{ movies1[,}\FunctionTok{c}\NormalTok{(}\StringTok{\textquotesingle{}duration\textquotesingle{}}\NormalTok{,}\StringTok{\textquotesingle{}avg\_vote\textquotesingle{}}\NormalTok{)]}
\FunctionTok{colnames}\NormalTok{(duration\_rating)}\OtherTok{\textless{}{-}}\FunctionTok{c}\NormalTok{(}\StringTok{"duration"}\NormalTok{,}\StringTok{"rating"}\NormalTok{)}
\NormalTok{df5 }\OtherTok{\textless{}{-}}\FunctionTok{arrange}\NormalTok{(duration\_rating, }\SpecialCharTok{{-}}\NormalTok{rating)}


\NormalTok{duration\_grouped }\OtherTok{\textless{}{-}}\NormalTok{ duration\_rating }\SpecialCharTok{\%\textgreater{}\%}
  \FunctionTok{group\_by}\NormalTok{(duration) }\SpecialCharTok{\%\textgreater{}\%}
    \FunctionTok{summarize}\NormalTok{(}
      \AttributeTok{mean\_vote =} \FunctionTok{mean}\NormalTok{(rating),}
      \AttributeTok{n=}\FunctionTok{n}\NormalTok{()}
\NormalTok{    )}

\FunctionTok{ggplot}\NormalTok{(duration\_grouped, }\FunctionTok{aes}\NormalTok{(}\AttributeTok{fill=}\NormalTok{mean\_vote, }\AttributeTok{y=}\NormalTok{mean\_vote, }\AttributeTok{x=}\NormalTok{duration)) }\SpecialCharTok{+}
    \FunctionTok{geom\_bar}\NormalTok{(}\AttributeTok{stat=}\StringTok{"identity"}\NormalTok{, }\AttributeTok{alpha=}\NormalTok{.}\DecValTok{8}\NormalTok{, }\AttributeTok{width=}\FloatTok{0.8}\NormalTok{) }\SpecialCharTok{+}
      \FunctionTok{scale\_fill\_fermenter}\NormalTok{(}\AttributeTok{palette =} \StringTok{"Set2"}\NormalTok{) }\SpecialCharTok{+}
        \FunctionTok{coord\_flip}\NormalTok{() }\SpecialCharTok{+}
            \FunctionTok{labs}\NormalTok{(}\AttributeTok{title=}\StringTok{"Ordered Bar Chart"}\NormalTok{,}
                \AttributeTok{subtitle=}\StringTok{"Mean rating for different film durations"}\NormalTok{,}
                \AttributeTok{caption=}\StringTok{"duration: Rating"}\NormalTok{)}
\end{Highlighting}
\end{Shaded}

\includegraphics{Midterm_final_word_files/figure-beamer/Q4_EDA_plot-1.pdf}
\end{block}

\begin{block}{Correlation Analysis}
\protect\hypertarget{correlation-analysis}{}
We can run a correlation analysis between the two continuous variables
to see if there is a relation between the duration of the movie and its
rating. The result of the Pearson's correlation test is as follows:

\begin{Shaded}
\begin{Highlighting}[]
\CommentTok{\#correlation }
\FunctionTok{cor.test}\NormalTok{(movies}\SpecialCharTok{$}\NormalTok{avg\_vote, }\FunctionTok{as.numeric}\NormalTok{(movies}\SpecialCharTok{$}\NormalTok{duration))}
\end{Highlighting}
\end{Shaded}

\begin{verbatim}
## 
##  Pearson's product-moment correlation
## 
## data:  movies$avg_vote and as.numeric(movies$duration)
## t = 73, df = 84111, p-value <2e-16
## alternative hypothesis: true correlation is not equal to 0
## 95 percent confidence interval:
##  0.238 0.250
## sample estimates:
##   cor 
## 0.244
\end{verbatim}

We see that there is a weak correlation between the two variables.
Plotting a linear regression line through the data points gives us the
following output:

\begin{Shaded}
\begin{Highlighting}[]
\FunctionTok{ggplot}\NormalTok{(movies,}\FunctionTok{aes}\NormalTok{(}\AttributeTok{x=}\NormalTok{duration, }\AttributeTok{y=}\NormalTok{avg\_vote)) }\SpecialCharTok{+} 
  \FunctionTok{scale\_fill\_fermenter}\NormalTok{(}\AttributeTok{palette =} \StringTok{"Set2"}\NormalTok{) }\SpecialCharTok{+}
    \FunctionTok{ylab}\NormalTok{(}\StringTok{\textquotesingle{}Rating\textquotesingle{}}\NormalTok{) }\SpecialCharTok{+}
      \FunctionTok{geom\_point}\NormalTok{() }\SpecialCharTok{+}
        \FunctionTok{geom\_smooth}\NormalTok{(}\AttributeTok{method=}\StringTok{\textquotesingle{}lm\textquotesingle{}}\NormalTok{) }\SpecialCharTok{+}
          \FunctionTok{ggtitle}\NormalTok{(}\StringTok{"Correlation plot between Duration and Average Rating"}\NormalTok{)}
\end{Highlighting}
\end{Shaded}

\includegraphics{Midterm_final_word_files/figure-beamer/Q4_corr_plot-1.pdf}
We can further analyze the relationship between the two variables by
introducing some transformations to the duration variable and see how it
affects the correlation coefficient.
\end{block}
\end{block}

\begin{block}{Effect of movie language on Average Vote}
\protect\hypertarget{effect-of-movie-language-on-average-vote}{}
We have solved this SMART question in a very similar way to the first
question. We split up the language column into its component languages
and then run hypothesis tests on them. Initial data exploration included
checking the mean movie ratings across different languages. We have
considered only those languages having more than 200 movies for
visualization purposes.

\begin{block}{Dataset Creation}
\protect\hypertarget{dataset-creation}{}
\begin{Shaded}
\begin{Highlighting}[]
\NormalTok{language\_df1 }\OtherTok{\textless{}{-}}\NormalTok{ movie\_ratings[,}\FunctionTok{c}\NormalTok{(}\StringTok{"language"}\NormalTok{, }\StringTok{"avg\_vote"}\NormalTok{)]}

\CommentTok{\# remove outliers}
\NormalTok{outliers }\OtherTok{\textless{}{-}} \FunctionTok{unique}\NormalTok{(}\FunctionTok{boxplot}\NormalTok{(language\_df1}\SpecialCharTok{$}\NormalTok{avg\_vote, }\AttributeTok{plot=}\ConstantTok{FALSE}\NormalTok{)}\SpecialCharTok{$}\NormalTok{out)}
\NormalTok{language\_df1 }\OtherTok{\textless{}{-}}\NormalTok{ language\_df1[}\SpecialCharTok{{-}}\FunctionTok{which}\NormalTok{(language\_df1}\SpecialCharTok{$}\NormalTok{avg\_vote }\SpecialCharTok{\%in\%}\NormalTok{ outliers),]}

\NormalTok{language\_df2 }\OtherTok{\textless{}{-}}\NormalTok{ language\_df1 }\SpecialCharTok{\%\textgreater{}\%}
  \FunctionTok{separate}\NormalTok{(language, }\FunctionTok{c}\NormalTok{(}\StringTok{"c1"}\NormalTok{,}\StringTok{\textquotesingle{}c2\textquotesingle{}}\NormalTok{,}\StringTok{\textquotesingle{}c3\textquotesingle{}}\NormalTok{,}\StringTok{"c4"}\NormalTok{,}\StringTok{\textquotesingle{}c5\textquotesingle{}}\NormalTok{,}\StringTok{\textquotesingle{}c6\textquotesingle{}}\NormalTok{),}\AttributeTok{sep =}\FunctionTok{c}\NormalTok{(}\StringTok{\textquotesingle{}, \textquotesingle{}}\NormalTok{))}

\NormalTok{df1 }\OtherTok{\textless{}{-}}\NormalTok{ language\_df2[,}\FunctionTok{c}\NormalTok{(}\StringTok{"c1"}\NormalTok{, }\StringTok{"avg\_vote"}\NormalTok{)] }\SpecialCharTok{\%\textgreater{}\%} \FunctionTok{drop\_na}\NormalTok{()}
\NormalTok{df2 }\OtherTok{\textless{}{-}}\NormalTok{ language\_df2[,}\FunctionTok{c}\NormalTok{(}\StringTok{"c2"}\NormalTok{, }\StringTok{"avg\_vote"}\NormalTok{)] }\SpecialCharTok{\%\textgreater{}\%} \FunctionTok{drop\_na}\NormalTok{()}
\NormalTok{df3 }\OtherTok{\textless{}{-}}\NormalTok{ language\_df2[,}\FunctionTok{c}\NormalTok{(}\StringTok{"c3"}\NormalTok{, }\StringTok{"avg\_vote"}\NormalTok{)] }\SpecialCharTok{\%\textgreater{}\%} \FunctionTok{drop\_na}\NormalTok{()}
\NormalTok{df4 }\OtherTok{\textless{}{-}}\NormalTok{ language\_df2[,}\FunctionTok{c}\NormalTok{(}\StringTok{"c4"}\NormalTok{, }\StringTok{"avg\_vote"}\NormalTok{)] }\SpecialCharTok{\%\textgreater{}\%} \FunctionTok{drop\_na}\NormalTok{()}
\NormalTok{df5 }\OtherTok{\textless{}{-}}\NormalTok{ language\_df2[,}\FunctionTok{c}\NormalTok{(}\StringTok{"c5"}\NormalTok{, }\StringTok{"avg\_vote"}\NormalTok{)] }\SpecialCharTok{\%\textgreater{}\%} \FunctionTok{drop\_na}\NormalTok{()}
\NormalTok{df6 }\OtherTok{\textless{}{-}}\NormalTok{ language\_df2[,}\FunctionTok{c}\NormalTok{(}\StringTok{"c6"}\NormalTok{, }\StringTok{"avg\_vote"}\NormalTok{)] }\SpecialCharTok{\%\textgreater{}\%} \FunctionTok{drop\_na}\NormalTok{()}

\FunctionTok{names}\NormalTok{(df1)[}\DecValTok{1}\NormalTok{] }\OtherTok{\textless{}{-}} \StringTok{"language"}
\FunctionTok{names}\NormalTok{(df2)[}\DecValTok{1}\NormalTok{] }\OtherTok{\textless{}{-}} \StringTok{"language"}
\FunctionTok{names}\NormalTok{(df3)[}\DecValTok{1}\NormalTok{] }\OtherTok{\textless{}{-}} \StringTok{"language"}
\FunctionTok{names}\NormalTok{(df4)[}\DecValTok{1}\NormalTok{] }\OtherTok{\textless{}{-}} \StringTok{"language"}
\FunctionTok{names}\NormalTok{(df5)[}\DecValTok{1}\NormalTok{] }\OtherTok{\textless{}{-}} \StringTok{"language"}
\FunctionTok{names}\NormalTok{(df6)[}\DecValTok{1}\NormalTok{] }\OtherTok{\textless{}{-}} \StringTok{"language"}

\NormalTok{language\_df }\OtherTok{\textless{}{-}} \FunctionTok{rbind}\NormalTok{(df1, df2, df3, df4, df5, df6)}
\end{Highlighting}
\end{Shaded}
\end{block}

\begin{block}{Data Visualization}
\protect\hypertarget{data-visualization-2}{}
\begin{Shaded}
\begin{Highlighting}[]
\FunctionTok{theme\_set}\NormalTok{(}\FunctionTok{theme\_bw}\NormalTok{())}

\NormalTok{language\_grouped }\OtherTok{\textless{}{-}}\NormalTok{ language\_df }\SpecialCharTok{\%\textgreater{}\%}
  \FunctionTok{group\_by}\NormalTok{(language) }\SpecialCharTok{\%\textgreater{}\%}
    \FunctionTok{filter}\NormalTok{(}\FunctionTok{n}\NormalTok{() }\SpecialCharTok{\textgreater{}=} \DecValTok{200}\NormalTok{) }\SpecialCharTok{\%\textgreater{}\%}
      \FunctionTok{summarize}\NormalTok{(}
        \AttributeTok{mean\_vote =} \FunctionTok{mean}\NormalTok{(avg\_vote),}
        \AttributeTok{n=}\FunctionTok{n}\NormalTok{()}
\NormalTok{      )}

\FunctionTok{ggplot}\NormalTok{(language\_grouped, }\FunctionTok{aes}\NormalTok{(}\AttributeTok{y=}\NormalTok{mean\_vote, }\AttributeTok{x=}\FunctionTok{reorder}\NormalTok{(language,mean\_vote))) }\SpecialCharTok{+}
    \FunctionTok{geom\_point}\NormalTok{(}\AttributeTok{width=}\NormalTok{.}\DecValTok{8}\NormalTok{, }\AttributeTok{stat=}\StringTok{"identity"}\NormalTok{) }\SpecialCharTok{+}
      \FunctionTok{scale\_fill\_fermenter}\NormalTok{(}\AttributeTok{palette =} \StringTok{"Set2"}\NormalTok{) }\SpecialCharTok{+}
          \FunctionTok{xlab}\NormalTok{(}\StringTok{"language"}\NormalTok{) }\SpecialCharTok{+}
          \FunctionTok{theme}\NormalTok{(}\AttributeTok{axis.text.x =} \FunctionTok{element\_text}\NormalTok{(}\AttributeTok{angle=}\DecValTok{45}\NormalTok{, }\AttributeTok{vjust=}\DecValTok{1}\NormalTok{, }\AttributeTok{hjust=}\DecValTok{1}\NormalTok{)) }\SpecialCharTok{+}
            \FunctionTok{labs}\NormalTok{(}\AttributeTok{title=}\StringTok{"Ordered Bar Chart"}\NormalTok{,}
                \AttributeTok{subtitle=}\StringTok{"Mean vote across film languages (\textgreater{}= 200 votes)"}\NormalTok{, }
                \AttributeTok{caption=}\StringTok{"language: mean\_vote"}\NormalTok{)}
\end{Highlighting}
\end{Shaded}

\includegraphics{Midterm_final_word_files/figure-beamer/Q5_visualization-1.pdf}
\end{block}

\begin{block}{Hypothesis Testing and P-value visualization}
\protect\hypertarget{hypothesis-testing-and-p-value-visualization-2}{}
We again run an ANOVA test followed by a post-hoc Tukey-Kramer test to
find language pairs that have significantly different average movie
ratings. The outputs for the tests are as follows:

\begin{Shaded}
\begin{Highlighting}[]
\NormalTok{language\_df\_filtered }\OtherTok{\textless{}{-}}\NormalTok{ language\_df }\SpecialCharTok{\%\textgreater{}\%}
  \FunctionTok{group\_by}\NormalTok{(language) }\SpecialCharTok{\%\textgreater{}\%}
    \FunctionTok{filter}\NormalTok{(}\FunctionTok{n}\NormalTok{() }\SpecialCharTok{\textgreater{}=} \DecValTok{30}\NormalTok{) }\SpecialCharTok{\%\textgreater{}\%}
      \FunctionTok{ungroup}\NormalTok{()}

\NormalTok{anovaRes }\OtherTok{=} \FunctionTok{aov}\NormalTok{(avg\_vote }\SpecialCharTok{\textasciitilde{}}\NormalTok{ language, }\AttributeTok{data=}\NormalTok{language\_df\_filtered)}
\NormalTok{anovaRes }
\end{Highlighting}
\end{Shaded}

\begin{verbatim}
## Call:
##    aov(formula = avg_vote ~ language, data = language_df_filtered)
## 
## Terms:
##                 language Residuals
## Sum of Squares      8880     98909
## Deg. of Freedom       75     93853
## 
## Residual standard error: 1.03
## Estimated effects may be unbalanced
\end{verbatim}

\begin{Shaded}
\begin{Highlighting}[]
\FunctionTok{summary}\NormalTok{(anovaRes)}
\end{Highlighting}
\end{Shaded}

\begin{verbatim}
##                Df Sum Sq Mean Sq F value Pr(>F)    
## language       75   8880   118.4     112 <2e-16 ***
## Residuals   93853  98909     1.1                   
## ---
## Signif. codes:  0 '***' 0.001 '**' 0.01 '*' 0.05 '.' 0.1 ' ' 1
\end{verbatim}

\begin{Shaded}
\begin{Highlighting}[]
\FunctionTok{names}\NormalTok{(anovaRes)}
\end{Highlighting}
\end{Shaded}

\begin{verbatim}
##  [1] "coefficients"  "residuals"     "effects"       "rank"         
##  [5] "fitted.values" "assign"        "qr"            "df.residual"  
##  [9] "contrasts"     "xlevels"       "call"          "terms"        
## [13] "model"
\end{verbatim}

\begin{Shaded}
\begin{Highlighting}[]
\NormalTok{tukeyLanguageAoV }\OtherTok{\textless{}{-}} \FunctionTok{TukeyHSD}\NormalTok{(anovaRes)}
\NormalTok{tukeyVar }\OtherTok{\textless{}{-}} \FunctionTok{data.frame}\NormalTok{(tukeyLanguageAoV}\SpecialCharTok{$}\NormalTok{language)}
\NormalTok{tukeyVarSig }\OtherTok{\textless{}{-}} \FunctionTok{subset}\NormalTok{(tukeyVar, p.adj }\SpecialCharTok{\textless{}=} \FloatTok{0.01}\NormalTok{)}
\FunctionTok{head}\NormalTok{(tukeyVarSig)}
\end{Highlighting}
\end{Shaded}

\begin{verbatim}
##                         diff     lwr    upr    p.adj
## English-Aboriginal    -0.885 -1.6442 -0.126 2.60e-03
## Azerbaijani-Afrikaans  1.207  0.4680  1.946 2.39e-08
## Bengali-Afrikaans      0.738  0.2570  1.220 3.95e-07
## Bulgarian-Afrikaans    0.596  0.0685  1.123 5.21e-03
## English-Afrikaans     -0.560 -0.9817 -0.139 7.34e-05
## Georgian-Afrikaans     0.783  0.2077  1.358 3.42e-05
\end{verbatim}

There are 76 languages in our dataset that have more than 30 movies in
its sample. The total number of pairs possible for these languages is
2850. The Tukey test returned a table of 840 pairs with significant
differences in their average movie ratings. Hence, we can say that the
movie language has a considerable effect on the movie votes.
\end{block}
\end{block}

\begin{block}{Voter demographics affecting average vote and votes}
\protect\hypertarget{voter-demographics-affecting-average-vote-and-votes}{}
We explored the relationship between the characteristics of the movie
viewers and the average vote of movies. From the histogram and the box
plot of the mean vote of movies across genders, we note that the average
rating by the male audience is slightly lower than in the female
audience. We used a t-test that showed a significant difference between
the average vote of males and females. The mean movie rating for the
male audience is 5.801412, while for the female audience is 6.133158.
Running a hypothesis test at 0.05 significance level as follows:

Ho -\textgreater{} The mean movie ratings given by different genders is
same

Ha -\textgreater{} The mean movie ratings given by different genders is
not same

\begin{block}{Average Vote vs.~Gender}
\protect\hypertarget{average-vote-vs.-gender}{}
\begin{Shaded}
\begin{Highlighting}[]
\CommentTok{\# rating}
\NormalTok{mean\_all }\OtherTok{\textless{}{-}} \FunctionTok{rowMeans}\NormalTok{(}\FunctionTok{subset}\NormalTok{(movie\_ratings[, }\FunctionTok{c}\NormalTok{(}\DecValTok{28}\NormalTok{, }\DecValTok{30}\NormalTok{, }\DecValTok{32}\NormalTok{)]))}
\NormalTok{mean\_male }\OtherTok{\textless{}{-}} \FunctionTok{rowMeans}\NormalTok{(}\FunctionTok{subset}\NormalTok{(movie\_ratings[, }\FunctionTok{c}\NormalTok{(}\DecValTok{36}\NormalTok{, }\DecValTok{38}\NormalTok{, }\DecValTok{40}\NormalTok{)]))}
\NormalTok{mean\_female }\OtherTok{\textless{}{-}} \FunctionTok{rowMeans}\NormalTok{(}\FunctionTok{subset}\NormalTok{(movie\_ratings[, }\FunctionTok{c}\NormalTok{(}\DecValTok{44}\NormalTok{, }\DecValTok{46}\NormalTok{, }\DecValTok{48}\NormalTok{)]))}

\NormalTok{movie\_gender }\OtherTok{=} \FunctionTok{data.frame}\NormalTok{(movie\_ratings[,}\FunctionTok{c}\NormalTok{(}\StringTok{"genre"}\NormalTok{)], mean\_all, mean\_male, mean\_female)}
\NormalTok{movie\_gender }\OtherTok{\textless{}{-}} \FunctionTok{aggregate}\NormalTok{(}\FunctionTok{list}\NormalTok{(movie\_gender}\SpecialCharTok{$}\NormalTok{mean\_all, movie\_gender}\SpecialCharTok{$}\NormalTok{mean\_male, movie\_gender}\SpecialCharTok{$}\NormalTok{mean\_female), }\AttributeTok{by =} \FunctionTok{list}\NormalTok{(movie\_gender}\SpecialCharTok{$}\StringTok{\textasciigrave{}}\AttributeTok{movie\_ratings...c..genre...}\StringTok{\textasciigrave{}}\NormalTok{), mean)}
\FunctionTok{colnames}\NormalTok{(movie\_gender) }\OtherTok{\textless{}{-}} \FunctionTok{c}\NormalTok{(}\StringTok{"genre"}\NormalTok{,}\StringTok{"all gender"}\NormalTok{,}\StringTok{"male"}\NormalTok{, }\StringTok{"female"}\NormalTok{)}

\CommentTok{\# t{-}test for movie rating}
\FunctionTok{t.test}\NormalTok{(movie\_gender}\SpecialCharTok{$}\NormalTok{male, movie\_gender}\SpecialCharTok{$}\NormalTok{female)}
\end{Highlighting}
\end{Shaded}

\begin{verbatim}
## 
##  Welch Two Sample t-test
## 
## data:  movie_gender$male and movie_gender$female
## t = -9, df = 2380, p-value <2e-16
## alternative hypothesis: true difference in means is not equal to 0
## 95 percent confidence interval:
##  -0.407 -0.257
## sample estimates:
## mean of x mean of y 
##      5.80      6.13
\end{verbatim}

\begin{Shaded}
\begin{Highlighting}[]
\CommentTok{\# barplot for movie ratings for both genders}
\FunctionTok{colnames}\NormalTok{(movie\_gender) }\OtherTok{\textless{}{-}} \FunctionTok{c}\NormalTok{(}\StringTok{"Genre"}\NormalTok{,}\StringTok{"All gender"}\NormalTok{,}\StringTok{"Male"}\NormalTok{, }\StringTok{"Female"}\NormalTok{)}
\FunctionTok{barplot}\NormalTok{(}\FunctionTok{colMeans}\NormalTok{(movie\_gender[,}\DecValTok{3}\SpecialCharTok{:}\DecValTok{4}\NormalTok{]), }\AttributeTok{col =} \FunctionTok{c}\NormalTok{(}\StringTok{"\#FC8D62"}\NormalTok{, }\StringTok{"\#8DA0CB"}\NormalTok{), }\AttributeTok{main=}\StringTok{"The average of movie rating in female and male"}\NormalTok{)}
\end{Highlighting}
\end{Shaded}

\includegraphics{Midterm_final_word_files/figure-beamer/Q6_movie_rating_vs_gender-1.pdf}

\begin{Shaded}
\begin{Highlighting}[]
\CommentTok{\# melt data frame into long}
\FunctionTok{colnames}\NormalTok{(movie\_gender) }\OtherTok{\textless{}{-}} \FunctionTok{c}\NormalTok{(}\StringTok{"genre"}\NormalTok{,}\StringTok{"all gender"}\NormalTok{,}\StringTok{"male"}\NormalTok{, }\StringTok{"female"}\NormalTok{)}
\NormalTok{movie\_gender }\OtherTok{\textless{}{-}} \FunctionTok{melt}\NormalTok{(movie\_gender)}

\CommentTok{\# boxplot: movie ratings vs. genders}
\FunctionTok{boxplot}\NormalTok{(value}\SpecialCharTok{\textasciitilde{}}\NormalTok{variable, }\AttributeTok{data=}\NormalTok{movie\_gender, }\AttributeTok{main=}\StringTok{"Movie rating vs. genders"}\NormalTok{, }\AttributeTok{xlab=}\StringTok{"Gender"}\NormalTok{, }\AttributeTok{ylab=}\StringTok{"Movie rating"}\NormalTok{, }\AttributeTok{col=}\FunctionTok{c}\NormalTok{(}\StringTok{"\#66C2A5"}\NormalTok{, }\StringTok{"\#FC8D62"}\NormalTok{, }\StringTok{"\#8DA0CB"}\NormalTok{)) }
\end{Highlighting}
\end{Shaded}

\includegraphics{Midterm_final_word_files/figure-beamer/Q6_movie_rating_vs_gender-2.pdf}
\end{block}

\begin{block}{Total Votes vs.~Gender}
\protect\hypertarget{total-votes-vs.-gender}{}
While analyzing the histogram on total movie votes by gender, the total
votes in the male audience are much higher than in the female audience.
Additionally, in the figure of the boxplot, it is shown that there are
outliers in both genders. After applying the t-test, we find a
significant difference between the movie votes in males and females.

Ho -\textgreater{} The total movie votes given by different genders is
same

Ha -\textgreater{} The total movie votes given by different genders is
not same

\begin{Shaded}
\begin{Highlighting}[]
\CommentTok{\# vote dataset with genders}
\NormalTok{vote\_all }\OtherTok{\textless{}{-}} \FunctionTok{subset}\NormalTok{(movie\_ratings[, }\FunctionTok{c}\NormalTok{(}\DecValTok{35}\NormalTok{,}\DecValTok{43}\NormalTok{)])}
\NormalTok{vote }\OtherTok{=} \FunctionTok{data.frame}\NormalTok{(movie\_ratings}\SpecialCharTok{$}\NormalTok{genre, vote\_all}\SpecialCharTok{$}\NormalTok{males\_allages\_votes, vote\_all}\SpecialCharTok{$}\NormalTok{females\_allages\_votes)}

\NormalTok{vote }\OtherTok{\textless{}{-}} \FunctionTok{aggregate}\NormalTok{(}\FunctionTok{list}\NormalTok{(vote}\SpecialCharTok{$}\NormalTok{vote\_all.males\_allages\_votes, vote}\SpecialCharTok{$}\NormalTok{vote\_all.females\_allages\_votes), }\AttributeTok{by =} \FunctionTok{list}\NormalTok{(vote}\SpecialCharTok{$}\NormalTok{movie\_ratings.genre), sum)}
\FunctionTok{colnames}\NormalTok{(vote) }\OtherTok{\textless{}{-}} \FunctionTok{c}\NormalTok{(}\StringTok{"Genre"}\NormalTok{,}\StringTok{"Male"}\NormalTok{, }\StringTok{"Female"}\NormalTok{)}


\CommentTok{\# t{-}test for movie votes}
\FunctionTok{t.test}\NormalTok{(vote}\SpecialCharTok{$}\NormalTok{Male, vote}\SpecialCharTok{$}\NormalTok{Female)}
\end{Highlighting}
\end{Shaded}

\begin{verbatim}
## 
##  Welch Two Sample t-test
## 
## data:  vote$Male and vote$Female
## t = 6, df = 1338, p-value = 2e-09
## alternative hypothesis: true difference in means is not equal to 0
## 95 percent confidence interval:
##  225994 441069
## sample estimates:
## mean of x mean of y 
##    431335     97803
\end{verbatim}

Difference in mean number of voters is significant between male and
female genders.

\begin{Shaded}
\begin{Highlighting}[]
\CommentTok{\# plot histogram plot for male}
\FunctionTok{hist}\NormalTok{(vote}\SpecialCharTok{$}\NormalTok{Male, }\AttributeTok{main =} \StringTok{"Histogram of the movie votes for male"}\NormalTok{, }\AttributeTok{xlab=}\StringTok{"Movie rating"}\NormalTok{, }\AttributeTok{col=}\StringTok{"\#FC8D62"}\NormalTok{, }\AttributeTok{breaks =} \DecValTok{15}\NormalTok{)}
\end{Highlighting}
\end{Shaded}

\includegraphics{Midterm_final_word_files/figure-beamer/movie_vote_vs_gender_plot-1.pdf}

\begin{Shaded}
\begin{Highlighting}[]
\CommentTok{\# plot qqplot for male}
\FunctionTok{qqnorm}\NormalTok{(vote}\SpecialCharTok{$}\NormalTok{Male, }\AttributeTok{main =} \StringTok{"Q{-}Q plot for movie votes for males"}\NormalTok{)}
\FunctionTok{qqline}\NormalTok{(vote}\SpecialCharTok{$}\NormalTok{Male)}
\end{Highlighting}
\end{Shaded}

\includegraphics{Midterm_final_word_files/figure-beamer/movie_vote_vs_gender_plot-2.pdf}

\begin{Shaded}
\begin{Highlighting}[]
\CommentTok{\# plot histogram plot for female}
\FunctionTok{hist}\NormalTok{(vote}\SpecialCharTok{$}\NormalTok{Female, }\AttributeTok{main =} \StringTok{"Histogram of the movie votes for female"}\NormalTok{, }\AttributeTok{xlab=}\StringTok{"Movie rating"}\NormalTok{, }\AttributeTok{col=}\StringTok{"\#8DA0CB"}\NormalTok{, }\AttributeTok{breaks =} \DecValTok{15}\NormalTok{)}
\end{Highlighting}
\end{Shaded}

\includegraphics{Midterm_final_word_files/figure-beamer/movie_vote_vs_gender_plot-3.pdf}

\begin{Shaded}
\begin{Highlighting}[]
\CommentTok{\# plot qqplot for female}
\FunctionTok{qqnorm}\NormalTok{(vote}\SpecialCharTok{$}\NormalTok{Female, }\AttributeTok{main =} \StringTok{"Q{-}Q plot for movie votes for females"}\NormalTok{)}
\FunctionTok{qqline}\NormalTok{(vote}\SpecialCharTok{$}\NormalTok{Female)}
\end{Highlighting}
\end{Shaded}

\includegraphics{Midterm_final_word_files/figure-beamer/movie_vote_vs_gender_plot-4.pdf}

\begin{Shaded}
\begin{Highlighting}[]
\CommentTok{\# bar plot: movie votes vs. genders}
\FunctionTok{barplot}\NormalTok{(}\FunctionTok{colSums}\NormalTok{(vote[,}\DecValTok{2}\SpecialCharTok{:}\DecValTok{3}\NormalTok{]), }\AttributeTok{col =} \FunctionTok{c}\NormalTok{(}\StringTok{"\#FC8D62"}\NormalTok{, }\StringTok{"\#8DA0CB"}\NormalTok{), }\AttributeTok{main=}\StringTok{"The total votes for females and males"}\NormalTok{)}
\end{Highlighting}
\end{Shaded}

\includegraphics{Midterm_final_word_files/figure-beamer/movie_vote_vs_gender_plot-5.pdf}

\begin{Shaded}
\begin{Highlighting}[]
\CommentTok{\# melt data frame into long}
\FunctionTok{colnames}\NormalTok{(vote) }\OtherTok{\textless{}{-}} \FunctionTok{c}\NormalTok{(}\StringTok{"genre"}\NormalTok{,}\StringTok{"male"}\NormalTok{, }\StringTok{"female"}\NormalTok{)}
\NormalTok{vote }\OtherTok{\textless{}{-}} \FunctionTok{melt}\NormalTok{(vote)}

\CommentTok{\# boxplot: genders vs. movie votes}
\FunctionTok{boxplot}\NormalTok{(value}\SpecialCharTok{\textasciitilde{}}\NormalTok{variable, }\AttributeTok{data=}\NormalTok{vote, }\AttributeTok{main=}\StringTok{"Movie votes vs. genders"}\NormalTok{, }\AttributeTok{xlab=}\StringTok{"Gender"}\NormalTok{,}\AttributeTok{ylab=}\StringTok{"Movie rating"}\NormalTok{,}\AttributeTok{col=}\FunctionTok{c}\NormalTok{(}\StringTok{"\#FC8D62"}\NormalTok{, }\StringTok{"\#8DA0CB"}\NormalTok{), }\AttributeTok{border=}\FunctionTok{c}\NormalTok{(}\StringTok{"\#FC8D62"}\NormalTok{, }\StringTok{"\#8DA0CB"}\NormalTok{))}
\end{Highlighting}
\end{Shaded}

\includegraphics{Midterm_final_word_files/figure-beamer/movie_vote_vs_gender_plot-6.pdf}

\begin{Shaded}
\begin{Highlighting}[]
\NormalTok{age\_df }\OtherTok{\textless{}{-}}\NormalTok{ movie\_ratings[,}\FunctionTok{c}\NormalTok{(}\StringTok{"allgenders\_18age\_avg\_vote"}\NormalTok{, }\StringTok{"allgenders\_30age\_avg\_vote"}\NormalTok{, }\StringTok{"allgenders\_45age\_avg\_vote"}\NormalTok{, }\StringTok{"avg\_vote"}\NormalTok{)]}
\FunctionTok{names}\NormalTok{(age\_df)[}\DecValTok{1}\SpecialCharTok{:}\DecValTok{3}\NormalTok{] }\OtherTok{\textless{}{-}} \FunctionTok{c}\NormalTok{(}\StringTok{"avg\_vote\_18"}\NormalTok{, }\StringTok{"avg\_vote\_30"}\NormalTok{, }\StringTok{"avg\_vote\_45"}\NormalTok{)}

\CommentTok{\# remove outliers}
\NormalTok{outliers\_18 }\OtherTok{\textless{}{-}} \FunctionTok{unique}\NormalTok{(}\FunctionTok{boxplot}\NormalTok{(age\_df}\SpecialCharTok{$}\NormalTok{avg\_vote\_18, }\AttributeTok{plot=}\ConstantTok{FALSE}\NormalTok{)}\SpecialCharTok{$}\NormalTok{out)}
\NormalTok{age\_df }\OtherTok{\textless{}{-}}\NormalTok{ age\_df[}\SpecialCharTok{{-}}\FunctionTok{which}\NormalTok{(age\_df}\SpecialCharTok{$}\NormalTok{avg\_vote\_18 }\SpecialCharTok{\%in\%}\NormalTok{ outliers\_18),]}

\NormalTok{outliers\_30 }\OtherTok{\textless{}{-}} \FunctionTok{unique}\NormalTok{(}\FunctionTok{boxplot}\NormalTok{(age\_df}\SpecialCharTok{$}\NormalTok{avg\_vote\_30, }\AttributeTok{plot=}\ConstantTok{FALSE}\NormalTok{)}\SpecialCharTok{$}\NormalTok{out)}
\NormalTok{age\_df }\OtherTok{\textless{}{-}}\NormalTok{ age\_df[}\SpecialCharTok{{-}}\FunctionTok{which}\NormalTok{(age\_df}\SpecialCharTok{$}\NormalTok{avg\_vote\_30 }\SpecialCharTok{\%in\%}\NormalTok{ outliers\_30),]}

\NormalTok{outliers\_45 }\OtherTok{\textless{}{-}} \FunctionTok{unique}\NormalTok{(}\FunctionTok{boxplot}\NormalTok{(age\_df}\SpecialCharTok{$}\NormalTok{avg\_vote\_45, }\AttributeTok{plot=}\ConstantTok{FALSE}\NormalTok{)}\SpecialCharTok{$}\NormalTok{out)}
\NormalTok{age\_df }\OtherTok{\textless{}{-}}\NormalTok{ age\_df[}\SpecialCharTok{{-}}\FunctionTok{which}\NormalTok{(age\_df}\SpecialCharTok{$}\NormalTok{avg\_vote\_45 }\SpecialCharTok{\%in\%}\NormalTok{ outliers\_45),]}

\NormalTok{gender\_df }\OtherTok{\textless{}{-}}\NormalTok{ movie\_ratings[,}\FunctionTok{c}\NormalTok{(}\StringTok{"males\_allages\_avg\_vote"}\NormalTok{, }\StringTok{"females\_allages\_avg\_vote"}\NormalTok{, }\StringTok{"avg\_vote"}\NormalTok{)]}
\FunctionTok{names}\NormalTok{(gender\_df)[}\DecValTok{1}\SpecialCharTok{:}\DecValTok{2}\NormalTok{] }\OtherTok{\textless{}{-}} \FunctionTok{c}\NormalTok{(}\StringTok{"avg\_vote\_male"}\NormalTok{, }\StringTok{"avg\_vote\_female"}\NormalTok{)}

\CommentTok{\# remove outliers}
\NormalTok{outliers\_male }\OtherTok{\textless{}{-}} \FunctionTok{unique}\NormalTok{(}\FunctionTok{boxplot}\NormalTok{(gender\_df}\SpecialCharTok{$}\NormalTok{avg\_vote\_male, }\AttributeTok{plot=}\ConstantTok{FALSE}\NormalTok{)}\SpecialCharTok{$}\NormalTok{out)}
\NormalTok{gender\_df }\OtherTok{\textless{}{-}}\NormalTok{ gender\_df[}\SpecialCharTok{{-}}\FunctionTok{which}\NormalTok{(gender\_df}\SpecialCharTok{$}\NormalTok{avg\_vote\_male }\SpecialCharTok{\%in\%}\NormalTok{ outliers\_male),]}

\NormalTok{outliers\_female }\OtherTok{\textless{}{-}} \FunctionTok{unique}\NormalTok{(}\FunctionTok{boxplot}\NormalTok{(gender\_df}\SpecialCharTok{$}\NormalTok{avg\_vote\_female, }\AttributeTok{plot=}\ConstantTok{FALSE}\NormalTok{)}\SpecialCharTok{$}\NormalTok{out)}
\NormalTok{gender\_df }\OtherTok{\textless{}{-}}\NormalTok{ gender\_df[}\SpecialCharTok{{-}}\FunctionTok{which}\NormalTok{(gender\_df}\SpecialCharTok{$}\NormalTok{avg\_vote\_female }\SpecialCharTok{\%in\%}\NormalTok{ outliers\_female),]}
\end{Highlighting}
\end{Shaded}
\end{block}

\begin{block}{Data Visualization}
\protect\hypertarget{data-visualization-3}{}
\begin{Shaded}
\begin{Highlighting}[]
\NormalTok{age\_df\_cor }\OtherTok{\textless{}{-}} \FunctionTok{cor}\NormalTok{(age\_df)}
\NormalTok{age\_df\_melt }\OtherTok{\textless{}{-}} \FunctionTok{melt}\NormalTok{(age\_df\_cor)}

\FunctionTok{ggplot}\NormalTok{(age\_df\_melt, }\FunctionTok{aes}\NormalTok{(}\AttributeTok{x=}\NormalTok{Var1, }\AttributeTok{y=}\NormalTok{Var2, }\AttributeTok{fill=}\NormalTok{value)) }\SpecialCharTok{+} 
  \FunctionTok{geom\_tile}\NormalTok{(}\FunctionTok{aes}\NormalTok{(}\AttributeTok{label =}\NormalTok{ value)) }\SpecialCharTok{+}
    \FunctionTok{geom\_text}\NormalTok{(}\FunctionTok{aes}\NormalTok{(}\AttributeTok{label =} \FunctionTok{round}\NormalTok{(value,}\DecValTok{2}\NormalTok{))) }\SpecialCharTok{+}
      \FunctionTok{theme}\NormalTok{(}\AttributeTok{axis.title.x=}\FunctionTok{element\_blank}\NormalTok{(), }\AttributeTok{axis.title.y=}\FunctionTok{element\_blank}\NormalTok{()) }\SpecialCharTok{+}
        \FunctionTok{scale\_fill\_gradient}\NormalTok{(}\AttributeTok{low=}\StringTok{"darkblue"}\NormalTok{, }\AttributeTok{high =} \StringTok{"lightblue"}\NormalTok{) }\SpecialCharTok{+}
          \FunctionTok{labs}\NormalTok{(}\AttributeTok{title=}\StringTok{"Heat Map"}\NormalTok{,}
              \AttributeTok{subtitle=}\StringTok{"Age Correlation Matrix"}\NormalTok{,}
              \AttributeTok{caption=}\StringTok{"age: avg\_vote"}\NormalTok{)}
\end{Highlighting}
\end{Shaded}

\includegraphics{Midterm_final_word_files/figure-beamer/Q6_visualization-1.pdf}

\begin{Shaded}
\begin{Highlighting}[]
\NormalTok{gender\_df\_cor }\OtherTok{\textless{}{-}} \FunctionTok{cor}\NormalTok{(gender\_df)}
\NormalTok{gender\_df\_melt }\OtherTok{\textless{}{-}} \FunctionTok{melt}\NormalTok{(gender\_df\_cor)}

\FunctionTok{ggplot}\NormalTok{(gender\_df\_melt, }\FunctionTok{aes}\NormalTok{(}\AttributeTok{x=}\NormalTok{Var1, }\AttributeTok{y=}\NormalTok{Var2, }\AttributeTok{fill=}\NormalTok{value)) }\SpecialCharTok{+} 
  \FunctionTok{geom\_tile}\NormalTok{() }\SpecialCharTok{+}
    \FunctionTok{geom\_text}\NormalTok{(}\FunctionTok{aes}\NormalTok{(}\AttributeTok{label =} \FunctionTok{round}\NormalTok{(value,}\DecValTok{2}\NormalTok{))) }\SpecialCharTok{+}
      \FunctionTok{theme}\NormalTok{(}\AttributeTok{axis.title.x=}\FunctionTok{element\_blank}\NormalTok{(), }\AttributeTok{axis.title.y=}\FunctionTok{element\_blank}\NormalTok{()) }\SpecialCharTok{+}
        \FunctionTok{scale\_fill\_fermenter}\NormalTok{(}\AttributeTok{palette =} \StringTok{"Blues"}\NormalTok{) }\SpecialCharTok{+}
          \FunctionTok{labs}\NormalTok{(}\AttributeTok{title=}\StringTok{"Heat Map"}\NormalTok{,}
              \AttributeTok{subtitle=}\StringTok{"Gender Correlation Matrix"}\NormalTok{,}
              \AttributeTok{caption=}\StringTok{"gender: avg\_vote"}\NormalTok{)}
\end{Highlighting}
\end{Shaded}

\includegraphics{Midterm_final_word_files/figure-beamer/Q6_visualization-2.pdf}

The age correlation matrix shows us that there is a weaker correlation
between age 45+ people and the rest of the viewers which could make them
a group of interest while building regression models. The gender
correlation matrix shows similar results with respect to female viewers.
\end{block}

\begin{block}{Hypothesis Testing}
\protect\hypertarget{hypothesis-testing-1}{}
Finally, we did a t-test to check for differences in the average vote of
three ranges of the audience age. The results show significant
differences in the average rating of the movie across different viewer
age ranges.

Ho -\textgreater{} The mean movie ratings given by different ages is
same

Ha -\textgreater{} The mean movie ratings given by different ages is not
same

\begin{Shaded}
\begin{Highlighting}[]
\NormalTok{ttest\_age\_18\_30 }\OtherTok{\textless{}{-}} \FunctionTok{t.test}\NormalTok{(age\_df}\SpecialCharTok{$}\NormalTok{avg\_vote\_18, age\_df}\SpecialCharTok{$}\NormalTok{avg\_vote\_30, }\AttributeTok{conf.level =} \FloatTok{0.99}\NormalTok{)}
\NormalTok{ttest\_age\_18\_30}
\end{Highlighting}
\end{Shaded}

\begin{verbatim}
## 
##  Welch Two Sample t-test
## 
## data:  age_df$avg_vote_18 and age_df$avg_vote_30
## t = 24, df = 1e+05, p-value <2e-16
## alternative hypothesis: true difference in means is not equal to 0
## 99 percent confidence interval:
##  0.125 0.156
## sample estimates:
## mean of x mean of y 
##      6.19      6.05
\end{verbatim}

\begin{Shaded}
\begin{Highlighting}[]
\NormalTok{ttest\_age\_30\_45 }\OtherTok{\textless{}{-}} \FunctionTok{t.test}\NormalTok{(age\_df}\SpecialCharTok{$}\NormalTok{avg\_vote\_30, age\_df}\SpecialCharTok{$}\NormalTok{avg\_vote\_45, }\AttributeTok{conf.level =} \FloatTok{0.99}\NormalTok{)}
\NormalTok{ttest\_age\_30\_45}
\end{Highlighting}
\end{Shaded}

\begin{verbatim}
## 
##  Welch Two Sample t-test
## 
## data:  age_df$avg_vote_30 and age_df$avg_vote_45
## t = 27, df = 1e+05, p-value <2e-16
## alternative hypothesis: true difference in means is not equal to 0
## 99 percent confidence interval:
##  0.136 0.164
## sample estimates:
## mean of x mean of y 
##      6.05      5.90
\end{verbatim}

\begin{Shaded}
\begin{Highlighting}[]
\NormalTok{ttest\_age\_45\_18 }\OtherTok{\textless{}{-}} \FunctionTok{t.test}\NormalTok{(age\_df}\SpecialCharTok{$}\NormalTok{avg\_vote\_45, age\_df}\SpecialCharTok{$}\NormalTok{avg\_vote\_18, }\AttributeTok{conf.level =} \FloatTok{0.99}\NormalTok{)}
\NormalTok{ttest\_age\_45\_18}
\end{Highlighting}
\end{Shaded}

\begin{verbatim}
## 
##  Welch Two Sample t-test
## 
## data:  age_df$avg_vote_45 and age_df$avg_vote_18
## t = -50, df = 1e+05, p-value <2e-16
## alternative hypothesis: true difference in means is not equal to 0
## 99 percent confidence interval:
##  -0.305 -0.276
## sample estimates:
## mean of x mean of y 
##      5.90      6.19
\end{verbatim}

\begin{Shaded}
\begin{Highlighting}[]
\NormalTok{ttest\_gender }\OtherTok{\textless{}{-}} \FunctionTok{t.test}\NormalTok{(gender\_df}\SpecialCharTok{$}\NormalTok{avg\_vote\_male, gender\_df}\SpecialCharTok{$}\NormalTok{avg\_vote\_female, }\AttributeTok{conf.level =} \FloatTok{0.99}\NormalTok{)}
\NormalTok{ttest\_gender}
\end{Highlighting}
\end{Shaded}

\begin{verbatim}
## 
##  Welch Two Sample t-test
## 
## data:  gender_df$avg_vote_male and gender_df$avg_vote_female
## t = -39, df = 1e+05, p-value <2e-16
## alternative hypothesis: true difference in means is not equal to 0
## 99 percent confidence interval:
##  -0.239 -0.210
## sample estimates:
## mean of x mean of y 
##      6.01      6.23
\end{verbatim}

From the first three tests, we see that the difference in average votes
is significant between ages 18, 30, and 45.

from the last test, we see that the difference in average votes is
significant between male and female genders.
\end{block}
\end{block}
\end{frame}

\begin{frame}{Conclusion}
\protect\hypertarget{conclusion}{}
Based on the massive movie information, our aim was to understand what
are the important factors that make a movie more successful than others.
So we analysed what kind of movies are more successful , in other words
get higher IMDB scores. We also showed the results of this analysis in
an intuitive way by visualizing the outcome using ggplot2 in R. Our
analysis states that all the factors that we considered in our smart
questions i.e, Genre, director, movie duration, movie language and voter
demographics have some amount of effect on the IMDB vote or score on a
movie.
\end{frame}

\begin{frame}{References}
\protect\hypertarget{references}{}
Bonferroni correction. from Wolfram MathWorld. (n.d.). Retrieved
November 9, 2021, from
\url{https://mathworld.wolfram.com/BonferroniCorrection.html}.

Brownlee, J. (2020, December 20). How to prepare movie review data for
sentiment analysis (text classification). Machine Learning Mastery.
Retrieved November 9, 2021, from
\url{https://machinelearningmastery.com/prepare-movie-review-data-sentiment-analysis/}.

Correlation (Pearson, Kendall, Spearman). Statistics Solutions. (2021,
August 10). Retrieved November 9, 2021, from
\url{https://www.statisticssolutions.com/free-resources/directory-of-statistical-analyses/correlation-pearson-kendall-spearman/}.

Hayes, A. (2021, October 25). T-test definition. Investopedia. Retrieved
November 9, 2021, from
\url{https://www.investopedia.com/terms/t/t-test.asp}.

Kenton, W. (2021, October 6). How analysis of variance (ANOVA) works.
Investopedia. Retrieved November 9, 2021, from
\url{https://www.investopedia.com/terms/a/anova.asp}.

Khan, A., Gul, M. A., Uddin, M. I., Ali Shah, S. A., Ahmad, S., Al
Firdausi, M. D., \& Zaindin, M. (2020, August 1). Summarizing online
movie reviews: A machine learning approach to big data analytics.
Scientific Programming. Retrieved November 9, 2021, from
\url{https://www.hindawi.com/journals/sp/2020/5812715/}.

Nickolas, S. (2021, October 22). What do correlation coefficients
positive, negative, and zero mean? Investopedia. Retrieved November 9,
2021, from
\url{https://www.investopedia.com/ask/answers/032515/what-does-it-mean-if-correlation-coefficient-positive-negative-or-zero.asp}.

Unofficialmerve. (2019, October 21). IMDB exploratory data analysis.
Kaggle. Retrieved November 9, 2021, from
\url{https://www.kaggle.com/unofficialmerve/imdb-exploratory-data-analysis/notebook}.
\end{frame}

\end{document}
